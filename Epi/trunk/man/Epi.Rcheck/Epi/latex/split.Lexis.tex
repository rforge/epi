\HeaderA{split.Lexis}{Split follow-up time in a Lexis object}{split.Lexis}
\aliasA{splitLexis}{split.Lexis}{splitLexis}
\keyword{manip}{split.Lexis}
\begin{Description}\relax
The \code{splitLexis} function divides each row of a \code{Lexis}
object into disjoint follow-up intervals according to the supplied
break points.
\end{Description}
\begin{Usage}
\begin{verbatim}
splitLexis(lex, breaks)
\end{verbatim}
\end{Usage}
\begin{Arguments}
\begin{ldescription}
\item[\code{lex}] an object of class \code{Lexis}
\item[\code{breaks}] a named list of numeric vectors giving the break points
on one or more time scales. Optionally, a single vector of
break points may be given, in which case the first time scale is assumed.

\end{ldescription}
\end{Arguments}
\begin{Value}
An object of class \code{Lexis} with multiple rows for each row of
the argument \code{lex}. Each row of the new \code{Lexis} object
contains the part of the follow-up interval that falls inside one of
the time bands defined by the break points.

The variables representing the various time scales, are appropriately
updated in the new \code{Lexis} object. The entry and exit status
variables are also updated according to the rule that the entry status
is retained until the end of follow-up. All other variables are
considered to represent variables that are constant in time, and so
are replicated across all rows having the same id value.
\end{Value}
\begin{Author}\relax
Martyn Plummer
\end{Author}
\begin{SeeAlso}\relax
\code{\LinkA{timeBand}{timeBand}}
\end{SeeAlso}

