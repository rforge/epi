\HeaderA{Lexis}{Create a Lexis object}{Lexis}
\keyword{survival}{Lexis}
\begin{Description}\relax
Create an object of class \code{Lexis} to represent follow-up on
multiple time scales.
\end{Description}
\begin{Usage}
\begin{verbatim}
Lexis(entry, exit, duration, entry.status = 0, exit.status = 0, id, data,
      merge=TRUE)
\end{verbatim}
\end{Usage}
\begin{Arguments}
\begin{ldescription}
\item[\code{entry}] a named list of entry times. Each element of the list is
a numeric variable representing the entry time on the named time
scale. All time scales must have the same units (e.g. years)
\item[\code{exit}] a named list of exit times
\item[\code{duration}] a numeric vector giving the duration of follow-up
\item[\code{entry.status}] a vector giving the status at entry
\item[\code{exit.status}] a vector giving status at exit. Any change in
status during follow-up is assumed to take place exactly at the exit time
\item[\code{id}] a vector giving a unique identity value for each row of the
Lexis object
\item[\code{data}] an optional data frame, list, or environment containing
the variables. If  not found in \code{data}, the variables are
taken from the environment from which \code{Lexis} was called
\item[\code{merge}] a logical flag. If \code{TRUE} then the \code{data} 
argument will be coerced to a data frame and then merged with
the resulting \code{Lexis} object
\end{ldescription}
\end{Arguments}
\begin{Details}\relax
The analysis of long-term population-based follow-up studies typically
requires multiple time scales to be taken into account, such as
age, calender time, or time since an event. A \code{Lexis} object is
a data frame with additional attributes that allows these multiple time
dimensions of follow-up to be managed.

Lexis objects are named after the German gemographer Wilhelm
Lexis (1837-1914), who is credited with the invention of the
"Lexis diagram" for representing population dynamics simultaneously
by age, period and cohort.

The \code{Lexis} function creates a minimal \code{Lexis} object with
only those variables required to define the follow-up history in each
row. Additional variables can be merged into the \code{Lexis} object
using the \code{merge} method for \code{Lexis} objects.
\end{Details}
\begin{Value}
An object of class \code{Lexis}. This is represented as a data frame
with a column for each time scale, and additional columns with the
following names:
\begin{ldescription}
\item[\code{lex.id}] Identification of the inidvidual
\item[\code{lex.deltat}] Duration of follow-up
\item[\code{lex.status1}] Entry status
\item[\code{lex.status2}] Exit status
\end{ldescription}

If \code{merge=TRUE} then the \code{Lexis} object will also contain
all variables from the \code{data} argument.
\end{Value}
\begin{Note}\relax
Only two of the three arguments \code{entry}, \code{exit} and
\code{duration} need to be given.  If the third parameter is missing,
it is imputed. If duration is imputed, it must be the same on
all time scales.
\end{Note}
\begin{Author}\relax
Martyn Plummer
\end{Author}
\begin{SeeAlso}\relax
\code{\LinkA{plot.Lexis}{plot.Lexis}}, \code{\LinkA{splitLexis}{splitLexis}},
\code{\LinkA{merge.Lexis}{merge.Lexis}}, \code{\LinkA{entry}{entry}},
\code{\LinkA{timeScales}{timeScales}},
\code{\LinkA{timeBand}{timeBand}}
\end{SeeAlso}

