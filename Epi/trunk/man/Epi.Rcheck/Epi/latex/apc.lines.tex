\HeaderA{apc.lines}{Plot APC-estimates in an APC-frame.}{apc.lines}
\keyword{hplot}{apc.lines}
\begin{Description}\relax
When an APC-frame has been produced by \code{\LinkA{apc.frame}{apc.frame}}, this
function draws a set of estimates from an APC-fit in the frame. An
optional drift parameter can be added to the period parameters and
subtracted from the cohort and age parameters.
\end{Description}
\begin{Usage}
\begin{verbatim}
 apc.lines( A, P, C,
        scale = c("log","ln","rates","inc","RR"),
    frame.par = NULL,
        drift = 0,
           c0 = median( C[,1] ),
           a0 = median( A[,1] ),
           p0 = c0 + a0,
           ci = rep( FALSE, 3 ),
          lwd = c(3,1,1),
          lty = 1,
          col = "black",
         type = "l",
        knots = FALSE,
          ... )
\end{verbatim}
\end{Usage}
\begin{Arguments}
\begin{ldescription}
\item[\code{A}] Age effects. A 4-column matrix with columns age, age-specific
rates, lower and upper c.i. If A is of class \code{apc} (see
\code{\LinkA{apc.fit}{apc.fit}}, \code{P}, \code{C}, \code{c0},
\code{a0} and \code{p0} are ignored, and the estimates from there
plotted.
\item[\code{P}] Period effects. Rate-ratios. Same form as for the age-effects.
\item[\code{C}] Cohort effects. Rate-ratios. Same form as for the age-effects.
\item[\code{scale}] Are effects given on a log-scale? Character variable, one
of \code{"log"}, \code{"ln"}, \code{"rates"}, \code{"inc"},
\code{"RR"}. If \code{"log"} or \code{"ln"} it is assumed that
effects are log(rates) and log(RRs) otherwise the actual effects are
assumed given in \code{A}, \code{P} and \code{C}. If \code{A} is of
class \code{apc}, it is assumed to be \code{"rates"}.
\item[\code{frame.par}] 2-element vector with the cohort-period offset and
RR multiplicator. This will typically be the result from the call of
\code{\LinkA{apc.frame}{apc.frame}}. See this for details.
\item[\code{drift}] The drift parameter to be added to the period effect. If
\code{scale="log"} this is assumed to be on the log-scale, otherwise
it is assumed to be a multiplicative factor per unit of the first
columns of  \code{A}, \code{P} and \code{C} 
\item[\code{c0}] The cohort where the drift is assumed to be 0; the subtracted
drift effect is \code{drift*(C[,1]-c0)}.
\item[\code{a0}] The age where the drift is assumed to be 0.
\item[\code{p0}] The period where the drift is assumed to be 0.
\item[\code{ci}] Should confidence interval be drawn. Logical or
character. If character, any occurrence of \code{"a"} or \code{"A"}
produces confidence intervals for the age-effect. Similarly for
period and cohort.
\item[\code{lwd}] Line widths for estimates, lower and upper confidence limits.
\item[\code{lty}] Linetypes for the three effects.
\item[\code{col}] Colours for the three effects.
\item[\code{type}] What type of lines / points should be used.
\item[\code{knots}] Should knots from the model be shown?
\item[\code{...}] Further parameters to be transmitted to \code{matlines} used
for plotting the three sets of curves.
\end{ldescription}
\end{Arguments}
\begin{Details}\relax
The drawing of three effects in an APC-frame is a rather trivial task,
and the main purpose of the utility is to provide a function that
easily adds the functionality of adding a drift so that several sets
of lines can be easily produced in the same frame.
\end{Details}
\begin{Value}
A list of three matrices with the effects plotted is
returned invisibly.
\end{Value}
\begin{Author}\relax
Bendix Carstensen, Steno Diabetes Center,
\url{http://www.pubhealth.ku.dk/~bxc}
\end{Author}
\begin{References}\relax
\end{References}
\begin{SeeAlso}\relax
\code{\LinkA{apc.frame}{apc.frame}, \\code\{\LinkA{apc.frame}{apc.frame}\}}
\end{SeeAlso}

