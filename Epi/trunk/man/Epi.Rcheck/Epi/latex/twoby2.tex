\HeaderA{twoby2}{Analysis of a two by two table}{twoby2}
\keyword{univar}{twoby2}
\keyword{htest}{twoby2}
\begin{Description}\relax
Computes the usual measures of association in a 2 by 2 table with
confidence intervals. Also produces asymtotic and exact tests. Assumes
that comparison of probability of the first column level between
levels of the row variable is of interest. Output requires that the
input matrix has meaningful row and column labels.
\end{Description}
\begin{Usage}
\begin{verbatim}
twoby2(exposure, outcome,
       alpha = 0.05, print = TRUE, dec = 4,
       conf.level = 1-alpha, F.lim = 10000)
\end{verbatim}
\end{Usage}
\begin{Arguments}
\begin{ldescription}
\item[\code{exposure}] If a table the analysis is based on the first two rows
and first two columns of this. If a variable, this variable is
tabulated against
\item[\code{outcome}] as the second variable
\item[\code{alpha}] Significance level
\item[\code{print}] Should the results be printed?
\item[\code{dec}] Number of decimals in the printout.
\item[\code{conf.level}] 1-\code{alpha}
\item[\code{F.lim}] If the table total exceeds \code{F.lim}, Fisher's exact
test is not computed
\end{ldescription}
\end{Arguments}
\begin{Value}
A list with elements:
\begin{ldescription}
\item[\code{table}] The analysed 2 x 2 table augmented with probabilities and
confidence intervals. The confidence intervals for the probabilities
are computed using the normal approximation to the
log-odds. Confidence intervals for the difference of proportions are
computed using method 10 from Newcombe, Stat.Med. 1998, 17, pp.873
ff.
\item[\code{measures}] A table of Odds-ratios and relative risk with
confidence intervals.
\item[\code{p.value}] Exact p-value for the null hypothesis of OR=1
\end{ldescription}
\end{Value}
\begin{Author}\relax
Mark Myatt. Modified by Bendix Carstensen.
\end{Author}
\begin{Examples}
\begin{ExampleCode}
Treat <- sample(c("A","B"), 50, rep=TRUE )
Resp <- c("Yes","No")[1+rbinom(50,1,0.3+0.2*(Treat=="A"))]
twoby2( Treat, Resp )                 
twoby2( table( Treat, Resp )[,2:1] ) # Comparison the other way round
\end{ExampleCode}
\end{Examples}

