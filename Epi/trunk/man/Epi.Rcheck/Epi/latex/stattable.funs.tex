\HeaderA{stattable.funs}{Special functions for use in stat.table}{stattable.funs}
\aliasA{count}{stattable.funs}{count}
\aliasA{percent}{stattable.funs}{percent}
\aliasA{ratio}{stattable.funs}{ratio}
\keyword{iteration}{stattable.funs}
\keyword{category}{stattable.funs}
\begin{Description}\relax
These functions may be used as \code{contents} arguments to the
function \code{stat.table}. They are defined internally in
\code{stat.table} and have no independent existence.
\end{Description}
\begin{Usage}
\begin{verbatim}
count(id)
ratio(d,y,scale=1, na.rm=TRUE)
percent(...)
\end{verbatim}
\end{Usage}
\begin{Arguments}
\begin{ldescription}
\item[\code{id}] numeric vector in which identical values identify the
same individual.
\item[\code{d, y}] numeric vectors of equal length (\code{d} for Deaths,
\code{y} for person-Years)
\item[\code{scale}] a scalar giving a value by which the ratio should be
multiplied
\item[\code{na.rm}] a logical value indicating whether \code{NA} values should
be stripped before computation proceeds.
\item[\code{...}] a list of variables taken from the \code{index} argument
to \code{\LinkA{stat.table}{stat.table}}
\end{ldescription}
\end{Arguments}
\begin{Value}
When used as a \code{contents} argument to \code{stat.table}, these
functions create the following tables:
\begin{ldescription}
\item[\code{this-is-escaped-codenormal-bracket30bracket-normal}] If given without argument (\code{count()}) it
returns a contingency table of counts. If given an \code{id}
argument it returns a table of the number of different values of
\code{id} in each cell, i.e. how many persons contribute in each
cell.
\item[\code{this-is-escaped-codenormal-bracket36bracket-normal}] returns a table of values
\code{scale * sum(d)/sum(y)}
\item[\code{this-is-escaped-codenormal-bracket40bracket-normal}] returns a table of percentages of the
classifying variables. Variables that are in the \code{index}
argument to \code{stat.table} but not in the call to
\code{percent} are used to define strata, within which the
percentages add up to 100.
\end{ldescription}

normal-bracket46bracket-normal
\end{Value}
\begin{Author}\relax
Martyn Plummer
\end{Author}
\begin{SeeAlso}\relax
\code{\LinkA{stat.table}{stat.table}}
\end{SeeAlso}

