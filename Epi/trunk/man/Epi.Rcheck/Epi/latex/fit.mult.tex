\HeaderA{fit.mult}{Fits a multiplicative relative risk model to interval censored data.}{fit.mult}
\keyword{models}{fit.mult}
\keyword{regression}{fit.mult}
\keyword{survival}{fit.mult}
\begin{Description}\relax
Utility function.

The model fitted assumes a piecewise constant baseline rate in
intervals specified by the argument \code{breaks}, and a
multiplicative relative risk function.
\end{Description}
\begin{Usage}
\begin{verbatim}
  fit.mult( y, rates.frame, cov.frame, start )
  \end{verbatim}
\end{Usage}
\begin{Arguments}
\begin{ldescription}
\item[\code{y}] Binary vector of outcomes
\item[\code{rates.frame}] Dataframe expanded from the original data by
\code{\LinkA{expand.data}{expand.data}}, cooresponding to covariates for the rate
parameters.
\item[\code{cov.frame}] do., but covariates corresponding to the
\code{formula} argument of \code{\LinkA{Icens}{Icens}}
\item[\code{start}] Starting values for the rate parameters. If not supplied,
then starting values are generated.
\end{ldescription}
\end{Arguments}
\begin{Details}\relax
The model is fitted by alternating between two generalized linear
models where one estimates the underlying rates in the intervals, and
the other estimates the log-relative risks.
\end{Details}
\begin{Value}
A list with three components:
\begin{ldescription}
\item[\code{rates}] A glm object from a binomial model with log-link,
estimating the baseline rates.
\item[\code{cov}] A glm object from a binomial model with complementary
log-log link, estimating the log-rate-ratios
\item[\code{niter}] Nuber of iterations, a scalar
\end{ldescription}
\end{Value}
\begin{Author}\relax
Martyn Plummer, \email{plummer@iarc.fr},
Bendix Carstensen, \email{bxc@steno.dk}
\end{Author}
\begin{References}\relax
B Carstensen: Regression models for interval censored
survival data: application to HIV infection in Danish homosexual
men. Statistics in Medicine, 15(20):2177-2189, 1996.

CP Farrington: Interval censored survival data: a generalized linear
modelling approach. Statistics in Medicine, 15(3):283-292, 1996.
\end{References}
\begin{SeeAlso}\relax
\code{\LinkA{Icens}{Icens}}
\code{\LinkA{fit.add}{fit.add}}
\end{SeeAlso}
\begin{Examples}
\begin{ExampleCode}
  data( HIV.dk ) 
  \end{ExampleCode}
\end{Examples}

