\HeaderA{isec}{Determine the intersection between follow-up intervals and a fixed
interval.}{isec}
\keyword{manip}{isec}
\keyword{datagen}{isec}
\begin{Description}\relax
For a given piece of follow-up for ach person, this function determines
the part of the follow-up which is inside a pre-specified interval. The
function may occasionally be useful but is mainly included as a tool to
be used in the function \code{\LinkA{Lexis}{Lexis}}.
\end{Description}
\begin{Usage}
\begin{verbatim}
  isec( enter, exit, fail = 0, int, cens.value = 0,
        Expand = 1:length(enter))
\end{verbatim}
\end{Usage}
\begin{Arguments}
\begin{ldescription}
\item[\code{enter}] Numereical vector of entry times.
\item[\code{exit}] Numerical vector of exit times.
\item[\code{fail}] Vector of exit status, i.e. status at time \code{exit}
\item[\code{int}] The fixed interval. Numerical vector of length 2.
\item[\code{cens.value}] Censoring value. Numerical. If the person survives the fixed
interval, this will be the value of the status indicator, \code{Fail}
\item[\code{Expand}] Person id to be carried to the output
\end{ldescription}
\end{Arguments}
\begin{Value}
A matrix with columns \code{Expand} - person identification,
\code{Enter} - date of entry into the interval, \code{Exit} - date of exit
from the interval and \code{Fail}.
\end{Value}
\begin{Author}\relax
Bendix Carstensen, Steno Diabetes Center,
\email{bxc@steno.dk}, \url{www.biostat.ku.dk/~bxc}
\end{Author}
\begin{SeeAlso}\relax
\code{\LinkA{Lexis}{Lexis}},
\code{\LinkA{icut}{icut}},
\code{\LinkA{fcut1}{fcut1}},
\code{\LinkA{fcut}{fcut}},
\code{\LinkA{ex1}{ex1}}
\end{SeeAlso}
\begin{Examples}
\begin{ExampleCode}
one <- round( runif( 15, 0, 15 ), 1 )
two <- round( runif( 15, 0, 15 ), 1 )
doe <- pmin( one, two )
dox <- pmax( one, two )
# Goofy data rows to test possibly odd behaviour
doe[1:3] <- dox[1:3] <- 8
dox[2] <- 6
dox[3] <- 7.5
# Some failure indicators
fail <- sample( 0:1, 15, replace=TRUE, prob=c(0.7,0.3) )
# So what have we got?
data.frame( doe, dox, fail )
# Find intersection with interval (4,8)
isec( doe, dox, fail, int=c(4,8) )
# See how it compares to original data
merge( data.frame( Expand=1:15, doe, dox, fail ),
       data.frame( isec( doe, dox, fail, int=c(4,8) ) ), all=TRUE )
  \end{ExampleCode}
\end{Examples}

