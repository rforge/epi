\HeaderA{merge.data.frame}{Merge data frame with a Lexis object}{merge.data.frame}
\keyword{ts}{merge.data.frame}
\begin{Description}\relax
Merge two data frames, or a data frame with a \code{Lexis} object.
\end{Description}
\begin{Usage}
\begin{verbatim}
## S3 method for class 'data.frame':
merge(x, y, ...)
\end{verbatim}
\end{Usage}
\begin{Arguments}
\begin{ldescription}
\item[\code{x, y}] data frames, or objects to be coerced into one
\item[\code{...}] optional arguments for the merge method
\end{ldescription}
\end{Arguments}
\begin{Details}\relax
This version of \code{merge.default} masks the one in the \code{base}.
It ensures that, if either \code{x} or \code{y} is a \code{Lexis}
object, then \code{merge.Lexis} is called.
\end{Details}
\begin{Value}
A merged \code{Lexis} object or data frame.
\end{Value}
\begin{Author}\relax
Martyn Plummer
\end{Author}
\begin{SeeAlso}\relax
\code{\LinkA{Lexis}{Lexis}}
\end{SeeAlso}

