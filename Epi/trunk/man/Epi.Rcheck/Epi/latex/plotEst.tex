\HeaderA{plotEst}{Plot estimates with confidence limits}{plotEst}
\aliasA{linesEst}{plotEst}{linesEst}
\aliasA{pointsEst}{plotEst}{pointsEst}
\keyword{hplot}{plotEst}
\keyword{models}{plotEst}
\begin{Description}\relax
Plots parameter estimates with confidence intervals, annotated with
parameter names. A dot is plotted at the estimate and a horizontal
line extending from the lower to the upper limit is superimposed.
\end{Description}
\begin{Usage}
\begin{verbatim}
plotEst( ests,
            y = dim(ests)[1]:1,
          txt = rownames(ests),
       txtpos = y, 
         ylim = range(y)-c(0.5,0),
         xlab = "",
         xtic = nice(ests[!is.na(ests)], log = xlog),
         xlim = range( xtic ),
         xlog = FALSE,
          pch = 16,
          cex = 1,
          lwd = 2,
          col = "black",
    col.lines = col,
   col.points = col,
         vref = NULL,
         grid = FALSE,
     col.grid = gray(0.9),
  restore.par = TRUE ) 

linesEst( ests, y = dim(ests)[1]:1, pch = 16, cex = 1, lwd = 2,
          col="black", col.lines=col, col.points=col )

pointsEst( ests, y = dim(ests)[1]:1, pch = 16, cex = 1, lwd = 2,
          col="black", col.lines=col, col.points=col )
\end{verbatim}
\end{Usage}
\begin{Arguments}
\begin{ldescription}
\item[\code{ests}] Matrix with three columns: Estimate, lower limit, upper
limit. If a model object is supplied, \code{\LinkA{ci.lin}{ci.lin}} is
invoked for this object first.
\item[\code{y}] Vertical position of the lines.
\item[\code{txt}] Annotation of the estimates.
\item[\code{txtpos}] Vertical position of the text. Defaults to \code{y}.
\item[\code{ylim}] Extent of the vertical axis.
\item[\code{xlab}] Annotation of the horizontal axis.
\item[\code{xtic}] Location of tickmarks on the x-axis.
\item[\code{xlim}] Extent of the x-axis.
\item[\code{xlog}] Should the x-axis be logarithmic?
\item[\code{pch}] What symbol should be used?
\item[\code{cex}] Expansion of the symbol.
\item[\code{col}] Colour of the points and lines.
\item[\code{col.lines}] Colour of the lines.
\item[\code{col.points}] Colour of the symbol.
\item[\code{lwd}] Thickness of the lines.
\item[\code{vref}] Where should vertical reference line(s) be drawn?
\item[\code{grid}] If TRUE, vertical gridlines are drawn at the
tickmarks. If a numerical vector is given vertical lines are drawn
at \code{grid}.
\item[\code{col.grid}] Colour of the vertical gridlines
\item[\code{restore.par}] Should the graphics parameters be restored? If set
to \code{FALSE} the coordinate system will still be available for
additional plotting, and \code{par("mai")} will still have the very
large value set in order to make room for the labelling of the
estimates.
\end{ldescription}
\end{Arguments}
\begin{Details}\relax
\code{plotEst} make a news plot, whereas \code{linesEst} and
\code{pointsEst} (identical functions) adds to an existing plot.
\end{Details}
\begin{Value}
NULL
\end{Value}
\begin{Author}\relax
Bendix Carstensen,
\email{bxc@steno.dk},
\url{http://www.pubhealth.ku.dk/~bxc}
\end{Author}
\begin{SeeAlso}\relax
ci.lin
\end{SeeAlso}
\begin{Examples}
\begin{ExampleCode}
# Bogus data and a linear model
f <- factor( sample( letters[1:5], 100, replace=TRUE ) )
x <- rnorm( 100 )
y <- 5 + 2 * as.integer( f ) + 0.8 * x + rnorm(100) * 2
m1 <- lm( y ~ f )

# Produce some confidence intervals for contrast to first level
( cf <- summary( m1 )$coef[2:5,1:2] %*% rbind( c(1,1,1), 1.96*(c(0,-1,1) ) ) )

# Plots with increasing amount of bells and whistles
par( mfcol=c(3,2), mar=c(3,3,2,1) )
plotEst( cf )
plotEst( cf, grid=TRUE )
plotEst( cf, grid=TRUE, cex=2, lwd=3 )
plotEst( cf, grid=TRUE, cex=2, col.points="red", col.lines="green" )
plotEst( cf, grid=TRUE, cex=2, col.points="red", col.lines="green",
          xlog=TRUE, xtic=c(1:8), xlim=c(0.8,6) )
rownames( cf )[1] <- "Contrast to fa:\n\n fb"
plotEst( cf, grid=TRUE, cex=2, col.points=rainbow(4), col.lines=rainbow(4), vref=1 )
  \end{ExampleCode}
\end{Examples}

