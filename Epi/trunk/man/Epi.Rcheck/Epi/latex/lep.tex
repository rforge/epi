\HeaderA{lep}{An unmatched case-control study of leprosy incidence}{lep}
\keyword{datasets}{lep}
\begin{Description}\relax
The \code{lep} data frame has 1370 rows and 7 columns. This was an
unmatched case-control study in which incident cases of leprosy in a
region of N. Malawi were compared with population controls.
\end{Description}
\begin{Format}\relax
This data frame contains the following columns:
\Tabular{rl}{
\code{id}: & subject identifier: a numeric vector \\
\code{d}: & case/control status: a numeric vector (1=case, 0=control) \\

\code{age}: & a factor with levels
\code{5-9} 
\code{10-14} 
\code{15-19} 
\code{20-24} 
\code{25-29} 
\code{30-44} 
\code{45+}  \\

\code{sex}: & a factor with levels
\code{male},  
\code{female}  \\ 

\code{bcg}: & presence of vaccine scar, a factor with levels
\code{no} 
\code{yes}  \\

\code{school}: & schooling, a factor with levels
\code{none} 
\code{1-5yrs} 
\code{6-8yrs} 
\code{sec/tert} \\

\code{house}: & housing, a factor with levels
\code{brick} 
\code{sunbrick} 
\code{wattle} 
\code{temp}  \\
}
\end{Format}
\begin{Source}\relax
The study is described in more detail in Clayton and Hills, Statistical
Models in Epidemiology, Oxford University Press, Oxford:1993.
\end{Source}
\begin{Examples}
\begin{ExampleCode}
data(lep)
\end{ExampleCode}
\end{Examples}

