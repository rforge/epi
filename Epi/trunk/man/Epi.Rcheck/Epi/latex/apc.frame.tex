\HeaderA{apc.frame}{Produce an empty frame for display of parameter-estimates from
Age-Period-Cohort-models.}{apc.frame}
\keyword{hplot}{apc.frame}
\begin{Description}\relax
A plot is generated where both the age-scale and the cohort/period
scale is on the x-axis. The left vertical axis will be a logarithmic
rate scale referring to age-effects and the right a logarithmic
rate-ratio scale of the same relative extent as the left referring to
the cohort and period effects (rate ratios).

Only an empty plot frame is generated. Curves or points must be added
with \code{points}, \code{lines} or the special utility function
\code{\LinkA{apc.lines}{apc.lines}}.
\end{Description}
\begin{Usage}
\begin{verbatim}
  apc.frame( a.lab,
            cp.lab,
             r.lab,
            rr.lab = r.lab / rr.ref,
            rr.ref = r.lab[length(r.lab)/2],
             a.tic = a.lab,
            cp.tic = cp.lab,
             r.tic = r.lab,
            rr.tic = r.tic / rr.ref,
           tic.fac = 1.3,
             a.txt = "Age",
            cp.txt = "Calendar time",
             r.txt = "Rate per 100,000 person-years",
            rr.txt = "Rate ratio",
               gap = diff(range(c(a.lab, a.tic)))/3,
          col.grid = gray(0.85),
             sides = c(1,2,4) )
\end{verbatim}
\end{Usage}
\begin{Arguments}
\begin{ldescription}
\item[\code{a.lab}] Numerical vector of labels for the age-axis.
\item[\code{cp.lab}] Numerical vector of labels for the cohort-period axis.
\item[\code{r.lab}] Numerical vector of labels for the rate-axis (left vertical)
\item[\code{rr.lab}] Numerical vector of labels for the RR-axis (right vertical)
\item[\code{rr.ref}] At what level of the rate scale is the RR=1 to be.
\item[\code{a.tic}] Location of additional tick marks on the age-scale
\item[\code{cp.tic}] Location of additional tick marks on the cohort-period-scale
\item[\code{r.tic}] Location of additional tick marks on the rate-scale
\item[\code{rr.tic}] Location of additional tick marks on the RR-axis.
\item[\code{tic.fac}] Factor with which to diminish intermediate tick marks
\item[\code{a.txt}] Text for the age-axis (left part of horizontal axis).
\item[\code{cp.txt}] Text for the cohort/period axis (right part of
horizontal axis).
\item[\code{r.txt}] Text for the rate axis (left vertical axis).
\item[\code{rr.txt}] Text for the rate-ratio axis (right vertical axis)
\item[\code{gap}] Gap between the age-scale and the cohort-period scale
\item[\code{col.grid}] Colour of the grid put in the plot.
\item[\code{sides}] Numerical vector indicating on which sides axes should
be drawn and annotated. This option is aimed for multi-panel
displays where axes only are put on the outer plots.
\end{ldescription}
\end{Arguments}
\begin{Details}\relax
The function produces an empty plot frame for display of results
from an age-period-cohort model, with age-specific rates in the left
side of the frame and cohort and period rate-ratio parameters in the
right side of the frame. There is a gap of \code{gap} between the
age-axis and the calendar time axis, vertical grid lines at
\code{c(a.lab,a.tic,cp.lab,cp.tic)}, and horizontal grid lines at
\code{c(r.lab,r.tic)}.

The function returns a numerical vector of
length 2, with names \code{c("cp.offset","RR.fac")}. The y-axis for
the plot will be a rate scale for the age-effects, and the x-axis will
be the age-scale. The cohort and period effects are plotted by
subtracting the first element (named \code{"cp.offset"}) of the returned result
form the cohort/period, and multiplying  the rate-ratios by the second
element of the returned result (named \code{"RR.fac"}).
\end{Details}
\begin{Value}
A numerical vector of length two, with names
\code{c("cp.offset","RR.fac")}. The first is the offset for the cohort
period-axis, the second the multiplication factor for the rate-ratio
scale.

Side-effect: A plot with axes and grid lines but no points or curves.
\end{Value}
\begin{Author}\relax
Bendix Carstensen, Steno Diabetes Center,
\url{http://www.pubhealth.ku.dk/~bxc/}
\end{Author}
\begin{References}\relax
\url{http://www.pubhealth.ku.dk/~bxc/APC/notes.pdf}
\end{References}
\begin{SeeAlso}\relax
\code{\LinkA{apc.lines}{apc.lines},\LinkA{apc.lines}{apc.lines}}
\end{SeeAlso}
\begin{Examples}
\begin{ExampleCode}
par( mar=c(4,4,1,4) )
res <-
apc.frame( a.lab=seq(30,90,20), cp.lab=seq(1880,2000,30), r.lab=c(1,2,5,10,20,50),
           a.tic=seq(30,90,10), cp.tic=seq(1880,2000,10), r.tic=c(1:10,1:5*10),
           gap=27 )
res
# What are the axes actually?
par(c("usr","xlog","ylog"))
# How to plot in the age-part: a point at (50,10)
points( 50, 10, pch=16, cex=2, col="blue" )
# How to plot in the cohort-period-part: a point at (1960,0.3)
points( 1960-res[1], 0.3*res[2], pch=16, cex=2, col="red" )
\end{ExampleCode}
\end{Examples}

