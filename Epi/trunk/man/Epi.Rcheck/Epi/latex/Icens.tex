\HeaderA{Icens}{Fits a regression model to interval censored data.}{Icens}
\aliasA{print.Icens}{Icens}{print.Icens}
\keyword{models}{Icens}
\keyword{regression}{Icens}
\keyword{survival}{Icens}
\begin{Description}\relax
The models fitted assumes a piecewise constant baseline rate in
intervals specified by the argument \code{breaks}, and for the
covariates either a multiplicative relative risk function (default) or
an additive excess risk function.
\end{Description}
\begin{Usage}
\begin{verbatim}
  Icens( first.well, last.well, first.ill,
         formula, model.type=c("MRR","AER"), breaks,
         boot=FALSE, alpha=0.05, keep.sample=FALSE,
         data )
  \end{verbatim}
\end{Usage}
\begin{Arguments}
\begin{ldescription}
\item[\code{first.well}] Time of entry to the study, i.e. the time first seen
without event. Numerical vector.
\item[\code{last.well}] Time last seen without event. Numerical vector.
\item[\code{first.ill}] Time first seen with event. Numerical vector.
\item[\code{formula}] Model formula for the log relative risk.
\item[\code{model.type}] Which model should be fitted.
\item[\code{breaks}] Breakpoints between intervals in which the underlying
timescale is assumed constant. Any observation outside the range of
\code{breaks} is discarded.
\item[\code{boot}] Should bootstrap be performed to produce confidence
intervals for parameters. If a number is given this will be the
number of bootsrap samples. The default is 1000.
\item[\code{alpha}] 1 minus the confidence level.
\item[\code{keep.sample}] Should the bootstrap sample of the parameter values
be returned?
\item[\code{data}] Data frame in which the times and formula are interpreted.
\end{ldescription}
\end{Arguments}
\begin{Details}\relax
The model is fitted by calling either \code{\LinkA{fit.mult}{fit.mult}} or
\code{\LinkA{fit.add}{fit.add}}.
\end{Details}
\begin{Value}
An object of class \code{"Icens"}: a list with three components:
\begin{ldescription}
\item[\code{rates}] A glm object from a binomial model with log-link,
estimating the baseline rates, and the excess risk if \code{"AER"}
is specfied.
\item[\code{cov}] A glm object from a binomial model with complementary
log-log link, estimating the log-rate-ratios. Only if \code{"MRR"}
is specfied.
\item[\code{niter}] Nuber of iterations, a scalar
\item[\code{boot.ci}] If \code{boot=TRUE}, a 3-column matrix with estimates
and 1-\code{alpha} confidence intervals for the parameters in the model.
\item[\code{sample}] A matrix of the parameterestimates from the
bootstrapping. Rows refer to parameters, columns to bootstrap samples.
\end{ldescription}
\end{Value}
\begin{Author}\relax
Martyn Plummer, \email{plummer@iarc.fr},
Bendix Carstensen, \email{bxc@steno.dk}
\end{Author}
\begin{References}\relax
B Carstensen: Regression models for interval censored
survival data: application to HIV infection in Danish homosexual
men. Statistics in Medicine, 15(20):2177-2189, 1996.

CP Farrington: Interval censored survival data: a generalized linear
modelling approach. Statistics in Medicine, 15(3):283-292, 1996.
\end{References}
\begin{SeeAlso}\relax
\code{\LinkA{fit.add}{fit.add}}
\code{\LinkA{fit.mult}{fit.mult}}
\end{SeeAlso}
\begin{Examples}
\begin{ExampleCode}
data( hivDK )
# Convert the dates to fractional years so that rates are
# expressed in cases per year
for( i in 2:4 ) hivDK[,i] <- cal.yr( hivDK[,i] )

m.RR <- Icens( entry, well, ill,
               model="MRR", formula=~pyr+us, breaks=seq(1980,1990,5),
               data=hivDK)
# Currently the MRR model returns a list with 2 glm objects.
round( ci.lin( m.RR$rates ), 4 )
round( ci.lin( m.RR$cov, Exp=TRUE ), 4 )
# There is actually a print method:
print( m.RR )

m.ER <- Icens( entry, well, ill,
               model="AER", formula=~pyr+us, breaks=seq(1980,1990,5),
               data=hivDK)
# There is actually a print method:
print( m.ER )
  \end{ExampleCode}
\end{Examples}

