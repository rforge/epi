\HeaderA{effx.match}{Function to calculate effects for individually matched case-control studies}{effx.match}
\keyword{models}{effx.match}
\keyword{regression}{effx.match}
\begin{Description}\relax
The function calculates the effects of an exposure on a response,
possibly stratified by a stratifying variable, and/or controlled for one
of more confounding variables.
\end{Description}
\begin{Usage}
\begin{verbatim}
effx.match(response,
exposure,
match,
strata=NULL,
control=NULL,
base=1,
sigdig=3,
alpha=0.05,
data=NULL) 
\end{verbatim}
\end{Usage}
\begin{Arguments}
\begin{ldescription}
\item[\code{response}] The \code{response} variable - must be numeric
\item[\code{exposure}] The \code{exposure} variable can be numeric or a factor
\item[\code{match}] The variable which identifies the matched sets
\item[\code{strata}] The \code{strata} stratifying variable - must be a factor
\item[\code{control}] The \code{control} variable(s). These are passed as a
list if there are more than one of them.
\item[\code{base}] Baseline for the effects of a categorical exposure, default 1
\item[\code{sigdig}] Number of significant digits for the effects, default 3
\item[\code{alpha}] 1 - confidence level
\item[\code{data}] \code{data} refers to the data used to evaluate the function
\end{ldescription}
\end{Arguments}
\begin{Details}\relax
Effects are calculated odds ratios.
The function is a wrapper for clogit, from the survival package.
The k-1 effects for a categorical  exposure with k levels are relative 
to a baseline which, by default, is the first level. The effect of a metric (quantitative) 
exposure is calculated per unit of exposure.
The exposure variable can be numeric or a factor, but if it is an ordered factor the order will be ignored.
\end{Details}
\begin{Value}
\begin{ldescription}
\item[\code{comp1 }] Effects of exposure
\item[\code{comp2 }] Tests of significance
\end{ldescription}
\end{Value}
\begin{Section}{Warning}
The function attaches the frame specified in the
\code{data=} argument, so there is a possibility that a variable in the
Global environment is masked by the attached frame. Watch out for this
warning in the output!
\end{Section}
\begin{Author}\relax
Michael Hills
\end{Author}
\begin{References}\relax
www.mhills.pwp.blueyonder.co.uk
\end{References}
\begin{Examples}
\begin{ExampleCode}
library(Epi)
library(survival)
data(bdendo)

# d is the case-control variable, set is the matching variable.
# The variable est is a factor and refers to estrogen use (yes,no)
# The variable age is numeric and refers to estrogen use (yes,no)
# effect of est on the odds of being a case
effx.match(d,exposure=est,match=set,data=bdendo)
# effect of age on the odds of being a case
effx.match(d,exposure=age,match=set,data=bdendo)
\end{ExampleCode}
\end{Examples}

