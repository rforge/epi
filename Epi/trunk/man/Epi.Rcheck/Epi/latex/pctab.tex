\HeaderA{pctab}{Create percentages in a table}{pctab}
\keyword{manip}{pctab}
\keyword{methods}{pctab}
\keyword{array}{pctab}
\begin{Description}\relax
Computes percentages and a margin of totals along a given margin of a table.
\end{Description}
\begin{Usage}
\begin{verbatim}
pctab(TT, margin = length(dim(TT)), dec=1)
\end{verbatim}
\end{Usage}
\begin{Arguments}
\begin{ldescription}
\item[\code{TT}] A table or array object
\item[\code{margin}] Which margin should be the the total?
\item[\code{dec}] How many decimals should be printed?
\end{ldescription}
\end{Arguments}
\begin{Value}
A table, where all dimensions except the one specified \code{margin}
has two extra levels named "All" (where all entries are 100) and "N".
The function prints the table with \code{dec} decimals.
\end{Value}
\begin{Author}\relax
Bendix Carstensen, Steno Diabetes Center,
\url{http://www.biostat.ku.dk/~bxc}
\end{Author}
\begin{SeeAlso}\relax
\code{\LinkA{addmargins}{addmargins}}
\end{SeeAlso}
\begin{Examples}
\begin{ExampleCode}
Aye <- sample( c("Yes","Si","Oui"), 177, replace=TRUE )
Bee <- sample( c("Hum","Buzz"), 177, replace=TRUE )
Sea <- sample( c("White","Black","Red","Dead"), 177, replace=TRUE )
A <- table( Aye, Bee, Sea )
A
ftable( pctab( A ) )
ftable( pctab( addmargins( A, 1 ), 3 ) )
round( ftable( pctab( addmargins( A, 1 ), 3 ), row.vars=3 ), 1)
\end{ExampleCode}
\end{Examples}

