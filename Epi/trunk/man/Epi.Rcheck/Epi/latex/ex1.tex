\HeaderA{ex1}{Split follow-up time along a timescale}{ex1}
\keyword{manip}{ex1}
\keyword{datagen}{ex1}
\begin{Description}\relax
Splits follow-up time at prespecified points of follow-up.
\end{Description}
\begin{Usage}
\begin{verbatim}
ex1( enter, exit, fail, origin = 0, scale = 1, breaks,
     data = data.frame(enter, exit, fail),
     Expand = 1:nrow(data) )
\end{verbatim}
\end{Usage}
\begin{Arguments}
\begin{ldescription}
\item[\code{enter}] Date of entry einto the study (start of follow-up). Numeric.
\item[\code{exit}] Date of termination of follow-up. Numeric.
\item[\code{fail}] Status at exit from the study.
\item[\code{origin}] Origin of the timescale to split on. Specified on the
input timescale, i.e. that of \code{enter} and \code{exit}.
\item[\code{scale}] Scaling between input and analysis timescale.
\item[\code{breaks}] Breakpoints on the analysis timescale. Follow-up before
\code{min(breaks)} and and ater \code{max(breks)} is discarded.
\item[\code{data}] Dataframe of variables to carry over to the output.
\item[\code{Expand}] Variable identifying original records.
\end{ldescription}
\end{Arguments}
\begin{Details}\relax
If \code{entry} and \code{exit} are given in days (for example as
\code{Date} variables, and we want follow-up cut at 5-year age
intervals, the we should choose \code{origin} equal to bithdate, scale
equal to 365.25 and breaks as \code{seq(0,100,5)}. Thus the input
timescale is calendar tiem measured in days, and output timescale is
age measured in years.
\end{Details}
\begin{Value}
A dataframe with one row per follow-up interval, and variables as in
\code{data}, preceded by the variables:
\begin{ldescription}
\item[\code{Expand}] Identification of the rows from the input dataframe.
\item[\code{Enter}] Entry date for the interval.
\item[\code{Exit}] Exit date for the interval.
\item[\code{Fail}] Failure indicator for end of the current interval.
\end{ldescription}
\end{Value}
\begin{Author}\relax
Bendix Carstensen, Steno Diabetes Center,
\email{bxc@steno.dk}, \url{www.biostat.ku.dk/~bxc}
\end{Author}
\begin{SeeAlso}\relax
\code{\LinkA{Lexis}{Lexis}},
\code{\LinkA{isec}{isec}},
\code{\LinkA{icut}{icut}},
\code{\LinkA{fcut1}{fcut1}},
\code{\LinkA{ex1}{ex1}}
\end{SeeAlso}
\begin{Examples}
\begin{ExampleCode}
one <- round( runif( 15, 0, 10 ), 1 )
two <- round( runif( 15, 0, 10 ), 1 )
doe <- pmin( one, two )
dox <- pmax( one, two )
# Goofy data rows to test possibly odd behaviour
doe[1:3] <- dox[1:3] <- 8
dox[2] <- 6
dox[3] <- 7.5
# Some failure indicators
fail <- sample( 0:1, 15, replace=TRUE, prob=c(0.7,0.3) )
# Split follow-up:
ex1( doe, dox, fail, breaks=0:10 )
\end{ExampleCode}
\end{Examples}

