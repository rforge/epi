\HeaderA{plotevent}{Plot Equivalence Classes}{plotevent}
\keyword{models}{plotevent}
\keyword{regression}{plotevent}
\keyword{survival}{plotevent}
\begin{Description}\relax
For interval censored data, segments of times between last.well and first.ill are plotted for each conversion in the data. It also plots the equivalence classes.
\end{Description}
\begin{Usage}
\begin{verbatim}
plotevent(last.well, first.ill, data)
\end{verbatim}
\end{Usage}
\begin{Arguments}
\begin{ldescription}
\item[\code{last.well}] Time at which the individuals are
last seen negative for the event 
\item[\code{first.ill}] Time at which the individuals are
first seen positive for the event 
\item[\code{data}] Data with a transversal shape 
\end{ldescription}
\end{Arguments}
\begin{Details}\relax
last.well and first.ill should be written as character in the function.
\end{Details}
\begin{Value}
Graph
\end{Value}
\begin{Author}\relax
Delphine Maucort-Boulch, Bendix Carstensen, Martyn Plummer
\end{Author}
\begin{References}\relax
Carstensen B. Regression models for interval censored survival data:
application to HIV infection in Danish homosexual men.Stat Med. 1996 Oct
30;15(20):2177-89. 

Lindsey JC, Ryan LM. Tutorial in biostatistics methods for
interval-censored data.Stat Med. 1998 Jan 30;17(2):219-38.
\end{References}
\begin{SeeAlso}\relax
\code{\LinkA{Icens}{Icens}}
\end{SeeAlso}

