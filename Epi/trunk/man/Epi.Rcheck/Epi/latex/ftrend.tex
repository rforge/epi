\HeaderA{ftrend}{Fit a floating trend to a factor in generalized linear model}{ftrend}
\keyword{regression}{ftrend}
\begin{Description}\relax
Fits a "floating trend" model to the given factor in a glm in a generalized
linear model by centering covariates.
\end{Description}
\begin{Usage}
\begin{verbatim}
ftrend(object, ...)
\end{verbatim}
\end{Usage}
\begin{Arguments}
\begin{ldescription}
\item[\code{object}] fitted \code{lm} or \code{glm} object. The model must not have an intercept term
\item[\code{...}] arguments to the \code{nlm} function
\end{ldescription}
\end{Arguments}
\begin{Details}\relax
\code{ftrend()} calculates "floating trend" estimates for factors in
generalized linear models.  This is an alternative to treatment
contrasts suggested by Greenland et al. (1999).  If a regression model
is fitted with no intercept term, then contrasts are  not used for the
first factor in the model. Instead, there is one parameter  for each
level of this factor.  However, the interpretation of these
parameters,  and their variance-covariance matrix, depends on the
numerical coding used for the  covariates. If an arbitrary constant is
added to the covariate values, then   the variance matrix is changed. 

The \code{ftrend()} function takes the fitted model and works out an optimal 
constant to add to the covariate values so that the covariance matrix is
approximately diagonal.  The parameter estimates can then be treated as
approximately independent, thus simplifying their presentation. This is
particularly useful for graphical display of dose-response relationships
(hence the name).

Greenland et al. (1999) originally suggested centring the covariates so that
their weighted mean, using the fitted weights from the model, is zero.  This
heuristic criterion is improved upon by \code{ftrend()} which uses the same 
minimum information divergence criterion as used by Plummer (2003) for
floating variance calculations. \code{ftrend()} calls \code{nlm()} to
do the minimization and will pass optional arguments to control it.
\end{Details}
\begin{Value}
A list with the following components
\begin{ldescription}
\item[\code{coef}] coefficients for model with adjusted covariates.
\item[\code{vcov}] Variance-covariance matrix of adjusted coefficients.
\end{ldescription}
\end{Value}
\begin{Note}\relax
The "floating trend" method is an alternative to the "floating
absolute risk" method, which is implemented in the function
\code{float()}.
\end{Note}
\begin{Author}\relax
Martyn Plummer
\end{Author}
\begin{References}\relax
Greenland S, Michels KB, Robins JM, Poole C and Willet WC (1999)
Presenting statistical uncertainty in trends and dose-response relations,
\emph{American Journal of Epidemiology}, \bold{149}, 1077-1086.
\end{References}
\begin{SeeAlso}\relax
\code{\LinkA{float}{float}}
\end{SeeAlso}

