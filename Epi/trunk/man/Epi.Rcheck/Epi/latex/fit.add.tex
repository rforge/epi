\HeaderA{fit.add}{Fit an addive excess risk model to interval censored data.}{fit.add}
\keyword{models}{fit.add}
\keyword{regression}{fit.add}
\keyword{survival}{fit.add}
\begin{Description}\relax
Utility function.

The model fitted assumes a piecewise constant inensity for the
baseline, and that the covariates act additively on the rate scale.
\end{Description}
\begin{Usage}
\begin{verbatim}
  fit.add( y, rates.frame, cov.frame, start )
  \end{verbatim}
\end{Usage}
\begin{Arguments}
\begin{ldescription}
\item[\code{y}] Binary vector of outcomes
\item[\code{rates.frame}] Dataframe expanded from the original data by
\code{\LinkA{expand.data}{expand.data}}, cooresponding to covariates for the rate
parameters.
\item[\code{cov.frame}] do., but covariates corresponding to the
\code{formula} argument of \code{\LinkA{Icens}{Icens}}
\item[\code{start}] Starting values for the rate parameters. If not supplied,
then starting values are generated.
\end{ldescription}
\end{Arguments}
\begin{Value}
A \code{\LinkA{glm}{glm}} object from a binomial model with log-link function.
\end{Value}
\begin{Author}\relax
Martyn Plummer, \email{plummer@iarc.fr}
\end{Author}
\begin{References}\relax
B Carstensen: Regression models for interval censored
survival data: application to HIV infection in Danish homosexual
men. Statistics in Medicine, 15(20):2177-2189, 1996.

CP Farrington: Interval censored survival data: a generalized linear
modelling approach. Statistics in Medicine, 15(3):283-292, 1996.
\end{References}
\begin{SeeAlso}\relax
\code{\LinkA{Icens}{Icens}}
\code{\LinkA{fit.mult}{fit.mult}}
\end{SeeAlso}
\begin{Examples}
\begin{ExampleCode}
  data( HIV.dk ) 
  \end{ExampleCode}
\end{Examples}

