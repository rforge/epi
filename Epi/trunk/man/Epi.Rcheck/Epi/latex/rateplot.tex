\HeaderA{rateplot}{Functions to plot rates from a table classified by age and
calendar time (period)}{rateplot}
\aliasA{Aplot}{rateplot}{Aplot}
\aliasA{Cplot}{rateplot}{Cplot}
\aliasA{Pplot}{rateplot}{Pplot}
\keyword{hplot}{rateplot}
\begin{Description}\relax
Produces plots of rates versus age, connected within period or cohort
(\code{Aplot}), rates versus period connected within age-groups
(\code{Pplot}) and rates and rates versus date of birth cohort
(\code{Cplot}). \code{rateplot} is a wrapper for these, allowing
to produce the four classical displays with a single call.
\end{Description}
\begin{Usage}
\begin{verbatim}
rateplot( rates,
          which = c("ap","ac","pa","ca"),
            age = as.numeric( dimnames( rates )[[1]] ),
            per = as.numeric( dimnames( rates )[[2]] ),
           grid = FALSE,
         a.grid = grid,
         p.grid = grid,
         c.grid = grid,
          ygrid = grid,
       col.grid = gray( 0.9 ),
          a.lim = range( age, na.rm=TRUE ) + c(0,diff( range( age ) )/30),
          p.lim = range( per, na.rm=TRUE ) + c(0,diff( range( age ) )/30),
          c.lim = NULL, 
           ylim = range( rates[rates>0], na.rm=TRUE ),
             at = NULL,
         labels = paste( at ),
          a.lab = "Age at diagnosis",
          p.lab = "Date of diagnosis",
          c.lab = "Date of birth",
           ylab = "Rates",
           type = "l",
            lwd = 2,
            lty = 1,
         log.ax = "y",
            las = 1,
            ann = FALSE,
          a.ann = ann,
          p.ann = ann,
          c.ann = ann,
          xannx = 1/20,
        cex.ann = 0.8,
         a.thin = seq( 1, length( age ), 2 ),
         p.thin = seq( 1, length( per ), 2 ),
         c.thin = seq( 2, length( age ) + length( per ) - 1, 2 ),
            col = par( "fg" ),
          a.col = col,
          p.col = col,
          c.col = col,
            ... )

Aplot( rates, age = as.numeric( dimnames( rates )[[1]] ),
       per = as.numeric( dimnames( rates )[[2]] ), grid = FALSE,
       a.grid = grid, ygrid = grid, col.grid = gray( 0.9 ),
       a.lim = range( age, na.rm=TRUE ), ylim = range( rates[rates>0], na.rm=TRUE ),
       at = NULL, labels = paste( at ), a.lab = names( dimnames( rates ) )[1],
       ylab = deparse( substitute( rates ) ), type = "l", lwd = 2, lty = 1,
       col = par( "fg" ), log.ax = "y", las = 1, c.col = col, p.col = col,
       c.ann = FALSE, p.ann = FALSE, xannx = 1/20, cex.ann = 0.8,
       c.thin = seq( 2, length( age ) + length( per ) - 1, 2 ),
       p.thin = seq( 1, length( per ), 2 ), p.lines = TRUE,
       c.lines = !p.lines, ... )

Pplot( rates, age = as.numeric( dimnames( rates )[[1]] ),
       per = as.numeric( dimnames( rates )[[2]] ), grid = FALSE,
       p.grid = grid, ygrid = grid, col.grid = gray( 0.9 ),
       p.lim = range( per, na.rm=TRUE ) + c(0,diff(range(per))/30),
       ylim = range( rates[rates>0], na.rm=TRUE ), p.lab = names( dimnames( rates ) )[2],
       ylab = deparse( substitute( rates ) ), at = NULL, labels = paste( at ),
       type = "l", lwd = 2, lty = 1, col = par( "fg" ), log.ax = "y",
       las = 1, ann = FALSE, cex.ann = 0.8, xannx = 1/20,
       a.thin = seq( 1, length( age ), 2 ), ... )

Cplot( rates, age = as.numeric( rownames( rates ) ),
       per = as.numeric( colnames( rates ) ), grid = FALSE,
       c.grid = grid, ygrid = grid, col.grid = gray( 0.9 ),
       c.lim = NULL, ylim = range( rates[rates>0], na.rm=TRUE ),
       at = NULL, labels = paste( at ), c.lab = names( dimnames( rates ) )[2],
       ylab = deparse( substitute( rates ) ), type = "l", lwd = 2, lty = 1,
       col = par( "fg" ), log.ax = "y", las = 1, xannx = 1/20, ann = FALSE,
       cex.ann = 0.8, a.thin = seq( 1, length( age ), 2 ), ...  )

\end{verbatim}
\end{Usage}
\begin{Arguments}
\begin{ldescription}
\item[\code{rates}] A two-dimensional table (or array) with rates to be plotted. It is
assumed that the first dimension is age and the second is period.
\item[\code{which}] A character vector with elements from
\code{c("ap","ac","apc","pa","ca")}, indication which plots should
be produced. One plot per element is produced. The first letter
indicates the x-axis of the plot, the remaining which groups
should be connected, i.e. \code{"pa"} will plot rates versus
period and connect age-classes, and \code{"apc"} will plot rates
versus age, and connect both periods and cohorts.
\item[\code{age}] Numerical vector giving the means of the
age-classes. Defaults to the rownames of \code{rates} as numeric.
\item[\code{per}] Numerical vector giving the means of the periods. Defaults
to the columnnames of \code{rates} as numeric.
\item[\code{grid}] Logical indicating whether a background grid should be drawn.
\item[\code{a.grid}] Logical indicating whether a background grid on the
age-axis should be drawn. If numerical it indicates the
age-coordinates of the grid.
\item[\code{p.grid}] do. for the period.
\item[\code{c.grid}] do. for the cohort.
\item[\code{ygrid}] do. for the rate-dimension.
\item[\code{col.grid}] The colour of the grid.
\item[\code{a.lim}] Range for the age-axis.
\item[\code{p.lim}] Range for the period-axis.
\item[\code{c.lim}] Range for the cohort-axis.
\item[\code{ylim}] Range for the y-axis (rates).
\item[\code{at}] Position of labels on the y-axis (rates).
\item[\code{labels}] Labels to put on the y-axis (rates).
\item[\code{a.lab}] Text on the age-axis. Defaults to "Age".
\item[\code{p.lab}] Text on the period-axis. Defaults to "Date of diagnosis".
\item[\code{c.lab}] Text on the cohort-axis. Defaults to "Date of birth".
\item[\code{ylab}] Text on the rate-axis. Defaults to the name of the rate-table.
\item[\code{type}] How should the curves be plotted. Defaults to \code{"l"}.
\item[\code{lwd}] Width of the lines. Defaults to 2.
\item[\code{lty}] Which type of lines should be used. Defaults to 1, a solid line.
\item[\code{log.ax}] Character with letters from \code{"apcyr"}, indicating
which axes should be logarithmic. \code{"y"} and \code{"r"} both
refer to the rate scale. Defaults to \code{"y"}.
\item[\code{las}] see \code{par}.
\item[\code{ann}] Should the curves be annotated?
\item[\code{a.ann}] Logical indicating whether age-curves should be annotated.
\item[\code{p.ann}] do. for period-curves.
\item[\code{c.ann}] do. for cohort-curves.
\item[\code{xannx}] The fraction that the x-axis is expanded when curves are annotated.
\item[\code{cex.ann}] Expansion factor for characters annotating curves.
\item[\code{a.thin}] Vector of integers indicating which of the age-classes
should be labelled.
\item[\code{p.thin}] do. for the periods.
\item[\code{c.thin}] do. for the cohorts.
\item[\code{col}] Colours for the curves.
\item[\code{a.col}] Colours for the age-curves.
\item[\code{p.col}] do. for the period-curves.
\item[\code{c.col}] do. for the cohort-curves.
\item[\code{p.lines}] Should rates from the same period be connected?
\item[\code{c.lines}] Should rates from the same cohort be connected?
\item[\code{...}] Additional arguments pssed on to \code{matlines} when
plotting the curves.
\end{ldescription}
\end{Arguments}
\begin{Details}\relax
Zero values of the rates are ignored. They are neiter in the plot nor in
the calculation of the axis ranges.
\end{Details}
\begin{Value}
\code{NULL}. The function is used for its side-effect, the plot.
\end{Value}
\begin{Author}\relax
Bendix Carstensen, Steno Diabetes Center,
\url{http://www.pubhealth.ku.dk/~bxc/}
\end{Author}
\begin{SeeAlso}\relax
\code{\LinkA{apc.frame}{apc.frame}}
\end{SeeAlso}
\begin{Examples}
\begin{ExampleCode}
data( blcaIT )
attach(blcaIT)

# Table of rates:
bl.rate <- tapply( D, list(age,period), sum ) /
           tapply( Y, list(age,period), sum )
bl.rate

# The four classical plots:
par( mfrow=c(2,2) )
rateplot( bl.rate*10^6 )

# The labels on the vertical axis could be nicer:
rateplot( bl.rate*10^6, at=10^(-1:3), labels=c(0.1,1,10,100,1000) ) 

# More bells an whistles
par( mfrow=c(1,3), mar=c(3,3,1,1), oma=c(0,3,0,0), mgp=c(3,1,0)/1.6 )
rateplot( bl.rate*10^6, ylab="", ann=TRUE, which=c("AC","PA","CA"),
                      at=10^(-1:3), labels=c(0.1,1,10,100,1000),
                      col=topo.colors(11), cex.ann=1.2 ) 
\end{ExampleCode}
\end{Examples}

