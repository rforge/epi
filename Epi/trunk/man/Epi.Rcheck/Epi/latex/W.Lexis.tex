\HeaderA{W.Lexis}{Split follow-up time in cohort studies.}{W.Lexis}
\keyword{manip}{W.Lexis}
\begin{Description}\relax
This is an old version of time-splitting. Consult the new version,
\code{\LinkA{Lexis}{Lexis}}.

For cohort input data the follow-up time is chopped in pieces along
several time scales, and a dataframe of follow-up intervals is
returned. Entry and exit times are assumed to be in the same timescale
(the input time scale).
\end{Description}
\begin{Usage}
\begin{verbatim}
W.Lexis( entry = 0,
          exit,
          fail,
        origin = 0,
         scale = 1,
        breaks,
       include = NULL,
          data = NULL )
\end{verbatim}
\end{Usage}
\begin{Arguments}
\begin{ldescription}
\item[\code{entry}] Date of entry on the input timescale. Numerical variable.
\item[\code{exit}] Date of exit on the input timescale. Numerical variable.
\item[\code{fail}] Failure indicator.
\item[\code{origin}] Origin of the output timescale(s) on the input
timescale. If for example the input timescale is calendar time and
the output timescale is (current) age, the the origin is date of
birth. If more than one timescale is used for splitting time
this is a list. Elements of the list must be named and must have the
same names as those in \code{scale} and \code{breaks}.
\item[\code{scale}] Scale of the output timescale(s) relative to the input
timescale. Elements of the list must be named and have the same
names as those in \code{origin} and \code{breaks}.
\item[\code{breaks}] Points on the output scale where the follow-up is
cut. If more than one timescale is used for splitting time this is a
list. Elements of the list must be named and must have the same
names as those in \code{origin} and \code{scale}.
\item[\code{include}] List of variables to carry unchanged from the original
dataframe to the output dataframe.
\item[\code{data}] Dataframe in which to interpret the arguments.
\end{ldescription}
\end{Arguments}
\begin{Details}\relax
The \code{data} is assumed to be a dataframe describing the follow-up
of a cohort, giving entry and exit time (on the input timescale) for each
individual as well as the exit status (failure status,
\code{fail}). The purpose of the function is to split each
individual's follow-up time along a number of timescales for example
age, calendar time, time since entry etc.
Any follow-up time before the first break
on any timescale or after the last break on any of these timescales
(the output timescales) is discarded.

NOTE: If a person has his/her exit before the first break or his entry
after the last break on any of the timescales the function will crash.
\end{Details}
\begin{Value}
A dataframe with one row per follow-up interval, with the following
variables: \begin{ldescription}
\item[\code{Expand}] A numerical vector with values in
\code{1:nrows(data)}, pointing at the rows of the input data frame
that is expanded to the output intervals.
\item[\code{Entry}] Date of entry for each interval. On the input time scale.
\item[\code{Exit}] Date of exit for each interval. On the input time scale.
\item[\code{Fail}] Exit status for each interval. Coded 0 for censoring, for
the last follow-up interval for each person it takes the value of
\code{fail}.
\item[\code{Time}] If \code{origin}, \code{scale} or \code{breaks} were given
as vectors this gives the left endpoints of the intervals
on the output scale.

If \code{origin}, \code{scale} or
\code{breaks}, were given as lists, there is no variable \code{Time}
in the dataframe, instead variables with the same names as
the list elements of these will be in the dataframe. The variables
have values corresponding to the left endpoints of the intervals on
the respective output time scales.
\item[\code{}] 
\end{ldescription}
 Finally, variables given in the argument \code{include}, values
replicated across all intervals from each individual.
\end{Value}
\begin{Author}\relax
David Clayton, approx. 2000.
Small modifications by Bendix Carstensen.
\end{Author}
\begin{References}\relax
This function has approximately the same functionality as
\code{stsplit} in Stata and the SAS-macro \code{\%Lexis}
(\url{http://www.biostat.ku.dk/~bxc/Lexis/Lexis.sas}). It has been
attempted to keep argument names similar between the three functions.
\end{References}
\begin{SeeAlso}\relax
\code{\LinkA{Lexis}{Lexis}},\code{\LinkA{Lexis.diagram}{Lexis.diagram}}
\end{SeeAlso}
\begin{Examples}
\begin{ExampleCode}
# A small bogus cohort
#
xcoh <- structure( list( id = c("A", "B", "C"),
                      birth = c("14/07/1952", "01/04/1954", "10/06/1987"),
                      entry = c("04/08/1965", "08/09/1972", "23/12/1991"),
                       exit = c("27/06/1997", "23/05/1995", "24/07/1998"),
                       fail = c(1, 0, 1) ),
                     .Names = c("id", "birth", "entry", "exit", "fail"),
                  row.names = c("1", "2", "3"),
                      class = "data.frame" )

# Convert the character dates into numerical variables (fractional years)
#
xcoh$bt <- cal.yr( xcoh$birth, format="%d/%m/%Y" )
xcoh$en <- cal.yr( xcoh$entry, format="%d/%m/%Y" )
xcoh$ex <- cal.yr( xcoh$exit , format="%d/%m/%Y" )

# See how it looks
#
xcoh 

# Split time along one time-axis
#
W.Lexis( entry = en,
          exit = ex,
          fail = fail,
         scale = 1,
        origin = bt,
        breaks = seq( 5, 40, 5 ),
       include = list( bt, en, ex, id ),
          data = xcoh )

# Split time along two time-axes
#
( x2 <- 
W.Lexis( entry = en,
          exit = ex,
          fail = fail,
         scale = 1,
        origin = list( per=0,                 age=bt          ),
        breaks = list( per=seq(1900,2000,10), age=seq(0,80,5) ),
       include = list( bt, en, ex, id ),
          data = xcoh ) )

# Tabulate the cases and the person-years
#
tapply( x2$Fail, list( x2$age, x2$per ), sum )
tapply( x2$Exit - x2$Entry, list( x2$age, x2$per ), sum )
\end{ExampleCode}
\end{Examples}

