\HeaderA{apc.plot}{Plot the estimates from a fitted Age-Period-Cohort model}{apc.plot}
\keyword{hplot}{apc.plot}
\begin{Description}\relax
This function plots the estimates created by \code{\LinkA{apc.fit}{apc.fit}} in a single 
graph. It just calls \code{\LinkA{apc.frame}{apc.frame}} after computing some sensible 
values of the parameters, and subsequently plots the estimates using 
\code{\LinkA{apc.lines}{apc.lines}}.
\end{Description}
\begin{Usage}
\begin{verbatim}
apc.plot(obj, r.txt = "Rate", ...)
\end{verbatim}
\end{Usage}
\begin{Arguments}
\begin{ldescription}
\item[\code{obj}] An object of class \code{apc}. 
\item[\code{r.txt}] The text to put on the vertical rate axis. 
\item[\code{...}] Additional arguments passed on to \code{\LinkA{apc.lines}{apc.lines}}. 
\end{ldescription}
\end{Arguments}
\begin{Value}
A numerical vector of length two, with names
\code{c("cp.offset","RR.fac")}. The first is the offset for the cohort
period-axis, the second the multiplication factor for the rate-ratio
scale. Therefore, if you want to plot at \code{(x,y)} in the right panel,
use \code{(x-res["cp.offset"],y/res["RR.fac"])=(x-res[1],y/res[2])}.
This vector should be supplied for the parameter \code{frame.par} to
\code{\LinkA{apc.lines}{apc.lines}} if more sets of estimates is plotted in the
same graph.
\end{Value}
\begin{Author}\relax
Bendix Carstensen, Steno Diabetes Center,
\url{http://www.pubhealth.ku.dk/~bxc}
\end{Author}
\begin{SeeAlso}\relax
\code{\LinkA{apc.lines}{apc.lines},\LinkA{apc.lines}{apc.lines},\LinkA{apc.lines}{apc.lines}}
\end{SeeAlso}
\begin{Examples}
\begin{ExampleCode}
data( lungDK )
attach( lungDK )
apc1 <- apc.fit( A=Ax, P=Px, D=D, Y=Y/10^5 )
fp <- apc.plot( apc1 )
apc.lines( apc1, frame.par=fp, drift=1.01, col="red" )
for( i in 1:11 )
  apc.lines( apc1, frame.par=fp, drift=1+(i-6)/100, col=rainbow(12)[i] )
\end{ExampleCode}
\end{Examples}

