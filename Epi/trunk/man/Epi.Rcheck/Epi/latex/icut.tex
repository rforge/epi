\HeaderA{icut}{Function to cut the follow-up in cohort at a point in time.}{icut}
\keyword{manip}{icut}
\keyword{datagen}{icut}
\begin{Description}\relax
The follow up time from \code{enter} to \code{exit} is classified as
to wheter it is before \code{cut} (\code{Time}=0) or after
\code{Time}=1). If \code{cut} is between \code{enter} and \code{exit},
the follow-up is split in two intervals, the first gets the value
\code{cens} for the status.
\end{Description}
\begin{Usage}
\begin{verbatim}
icut( enter, exit, cut, fail = 0, cens.value = 0,
      data = data.frame(enter, exit, fail, cut),
      Expand = 1:nrow( data ),
      na.cut = Inf )
\end{verbatim}
\end{Usage}
\begin{Arguments}
\begin{ldescription}
\item[\code{enter}] Date of entry. Numerical vector.
\item[\code{exit}] Date of exit. Numerical vector.
\item[\code{fail}] Indicator if exit status.
\item[\code{cens.value}] Value to use for censoring status.
\item[\code{cut}] Date where to cut follow-up. Numerical vector.
\item[\code{data}] Dataframe of variables to carry over to the output dataframe.
\item[\code{Expand}] Variable identifying original records.
\item[\code{na.cut}] What value should be assigned to missing values of the
cutpoint. Defaults to \code{Inf}, so the inetrmediate event is
considered not to have occcurred. If set to \code{-Inf}, all persons
with missing \code{cut} are considered to have had an intermediate
event. If set to \code{NA} records with missing \code{cut} are
omitted from the result.
\end{ldescription}
\end{Arguments}
\begin{Details}\relax
The purpose of this function is to divide follow-up into pre- and post
some intermediate event like recurrence of disease, thus enabling
Follow-up for persons with a recurrence date (\code{cut}) will be
split in two, with indication (in \code{Time}) of what is pre and what
is post recurrence. This is typically what precedes a survival
analysis where recurrence is modelled as a time-dependent variable.
\end{Details}
\begin{Value}
A data frame with one row per interval of follow up and columns given
in the \code{data} argument, preceded by the columns:
\begin{ldescription}
\item[\code{Expand}] Identification of the rows from the input dataframe.
\item[\code{Enter}] Entry date for the interval.
\item[\code{Exit}] Exit date for the interval.
\item[\code{Fail}] Failure indicator for the interval.
\item[\code{Time}] Indicator variable for intervals after \code{cut}.
\end{ldescription}
\end{Value}
\begin{Author}\relax
Bendix Carstensen, Steno Diabetes Center,
\email{bxc@steno.dk}, \url{www.biostat.ku.dk/~bxc}
\end{Author}
\begin{SeeAlso}\relax
\code{\LinkA{Lexis}{Lexis}},
\code{\LinkA{isec}{isec}},
\code{\LinkA{fcut1}{fcut1}},
\code{\LinkA{fcut}{fcut}},
\code{\LinkA{ex1}{ex1}}
\end{SeeAlso}
\begin{Examples}
\begin{ExampleCode}
one <- round( runif( 15, 0, 15 ), 1 )
two <- round( runif( 15, 0, 15 ), 1 )
doe <- pmin( one, two )
dox <- pmax( one, two )
# Goofy data rows to test possibly odd behaviour
doe[1:3] <- dox[1:3] <- 8
dox[2] <- 6
dox[3] <- 7.5
# Some failure indicators
fail <- sample( 0:1, 15, replace=TRUE, prob=c(0.7,0.3) )
# So what have we got
data.frame( doe, dox, fail )
# Cut follow-up at 5
icut( doe, dox, fail, cut=5 )
\end{ExampleCode}
\end{Examples}

