\HeaderA{subset.Lexis}{Subsetting Lexis objects}{subset.Lexis}
\keyword{manip}{subset.Lexis}
\begin{Description}\relax
Return subsets of Lexis objects which meet conditions
\end{Description}
\begin{Usage}
\begin{verbatim}
## S3 method for class 'Lexis':
subset(x, ...)
\end{verbatim}
\end{Usage}
\begin{Arguments}
\begin{ldescription}
\item[\code{x}] an object of class \code{Lexis}
\item[\code{...}] additional arguments to be passed to \code{subset.data.frame}
\end{ldescription}
\end{Arguments}
\begin{Details}\relax
The subset method for \code{Lexis} objects works exactly as the
method for data frames.
\end{Details}
\begin{Value}
A \code{Lexis} object with selected rows and columns.
\end{Value}
\begin{Author}\relax
Martyn Plummer
\end{Author}
\begin{SeeAlso}\relax
\code{\LinkA{Lexis}{Lexis}}, \code{\LinkA{merge.Lexis}{merge.Lexis}}
\end{SeeAlso}

