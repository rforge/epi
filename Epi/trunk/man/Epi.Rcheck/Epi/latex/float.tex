\HeaderA{float}{Calculate floated variances}{float}
\aliasA{print.floated}{float}{print.floated}
\keyword{regression}{float}
\begin{Description}\relax
Given a fitted  model object, the \code{float()} function calculates
floating variances (aka quasi-variances) for a given factor in the model.
\end{Description}
\begin{Usage}
\begin{verbatim}
float(object, factor, iter.max=50)
\end{verbatim}
\end{Usage}
\begin{Arguments}
\begin{ldescription}
\item[\code{object}] a fitted model object
\item[\code{factor}] character string giving the name of the factor of
interest. If this is not given, the first factor in the model is used.
\item[\code{iter.max}] Maximum number of iterations for EM algorithm
\end{ldescription}
\end{Arguments}
\begin{Details}\relax
float() implements the "floating absolute risk" proposal of Easton,
Peto and Babiker(1992). This is an alternative way of presenting
parameter estimates for factors in regression models, which avoids
some of the difficulties of treatment contrasts. It was originally
designed for epidemiological studies of relative risk (hence the name)
but the idea is widely applicable.  

A problem with treatment contrasts is that they are not
orthogonal. The variances of the treatment contrasts may be inflated by
a poor choice of reference level, and the correlations between them
may be very high.  float() associates each level of the factor,
including the reference level, with a"floating" variance (or
quasi-variance). Floating variances are not real variances, but
they can be used to calculate the variance of any contrast by treating
each level as independent.

Plummer (2003) showed that floating variances can be derived from a
covariance structure model applied to the variance-covariance
matrix of the parameter estimates. This model can be fitted by
minimizing the Kullback-Leibler information divergence between the
true and distributions for the parameter estimates and the
distribution given by the covariance structure model. Fitting is
done using the EM algorithm.

In order to check the goodness-of-fit of the floating variance
model, \code{float()} compares the standard errors predicted
by the model with the standard errors derived from the true
variance-covariance matrix of the parameter contrasts. The maximum
and minimum ratios between true and model standard errors are
calculated over all possible contrasts. These should be within 5
percent, or the use of the floating variances may lead to invalid
confidence intervals.
\end{Details}
\begin{Value}
An object of class \code{floated}. This is a list with the following
components
\begin{ldescription}
\item[\code{coef}] vector of coefficients. These are the same as the
treatment contrasts but the reference level is present with
coefficient 0.
\item[\code{var}] vector of floating (or quasi-) variances
\item[\code{limits}] Bounds on the accuracy of standard errors over all
possible contrasts
\end{ldescription}
\end{Value}
\begin{Note}\relax
Menezes(1999) and Firth and Menezes (2004) take a slightly different
approach to this problem, using a pseudo-likelihood approach to fit
the quasi-variance model. Their work is implemented in the package qvcalc.
\end{Note}
\begin{Author}\relax
Martyn Plummer
\end{Author}
\begin{References}\relax
Easton DF, Peto J and Babiker GAG (1991) Floating absolute risk: An
alternative to relative risk in survival and case control analysis
avoiding an arbitrary reference group. \emph{Statistics in Medicine},
\bold{10}, 1025-1035.

Firth D and Mezezes RX (2004)  Quasi-variances.
\emph{Biometrika} \bold{91}, 65-80.

Menezes RX(1999)  More useful standard errors for group and factor
effects in generalized linear models.  \emph{D.Phil. Thesis},
Department of Statistics, University of Oxford.

Plummer M (2003) Improved estimates of floating absolute risk,
\emph{Statistics in Medicine}, \bold{23}, 93-104.
\end{References}
\begin{SeeAlso}\relax
\code{\LinkA{ftrend}{ftrend}}, \code{qvcalc}
\end{SeeAlso}

