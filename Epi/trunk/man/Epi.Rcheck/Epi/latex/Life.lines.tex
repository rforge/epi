\HeaderA{Life.lines}{Compute dates/ages for life lines in a Lexis diagram}{Life.lines}
\keyword{manip}{Life.lines}
\keyword{dplot}{Life.lines}
\begin{Description}\relax
Fills out the missing information for follow up of persons in a Lexis
diagram if sufficient information is given.
\end{Description}
\begin{Usage}
\begin{verbatim}
Life.lines( entry.date = NA,
             exit.date = NA,
            birth.date = NA,
             entry.age = NA,
              exit.age = NA,
             risk.time = NA )
\end{verbatim}
\end{Usage}
\begin{Arguments}
\begin{ldescription}
\item[\code{entry.date, exit.date,birth.date, entry.age, exit.age,
risk.time}] Vectors defining lifelines to be plotted 
in the diagram. At least three must be given to produce a result.
Not all subsets of three will suffice, the given subset has to 
define life lines. If insufficient data is given, nothing is
returned and a warning is given.
\end{ldescription}
\end{Arguments}
\begin{Value}
Data frame with variables \code{entry.date}, \code{entry.age},
\code{exit.date}, \code{exit.age}, \code{risk.time},
\code{birth.date}, with all entries computed for each person. If any
of \code{entry.date}, \code{exit.date} or \code{birth.date} are of
class \code{Date} or if any of \code{entry.age}, \code{exit.age} or
\code{risk.time} are of class \code{difftime} the date variables will
be of class  \code{Date} and the other three of class
\code{difftime}.
\end{Value}
\begin{SeeAlso}\relax
\code{\LinkA{Lexis.diagram}{Lexis.diagram}},
\code{\LinkA{Lexis.lines}{Lexis.lines}}
\end{SeeAlso}
\begin{Examples}
\begin{ExampleCode}
( Life.lines( entry.age = c(3,30,45),
              risk.time = c(25,5,14),
             birth.date = c(1970,1931,1925.7) ) )

# Draw a Lexis diagram
Lexis.diagram()

# Compute entry and exit age and date.
( LL <-  Life.lines( entry.age = c(3,30,45),
                     risk.time = c(25,5,14),
                    birth.date = c(1970,1931,1925.7) ) )
segments( LL[,1], LL[,2], LL[,3], LL[,4] ) # Plot the life lines.

# Compute entry and exit age and date, supplying a date variable
bd <- ( c(1970,1931,1925.7) - 1970 ) * 365.25
class( bd ) <- "Date"
( Life.lines( entry.age = c(3,30,45),
              risk.time = c(25,5,14),
             birth.date = bd ) )
\end{ExampleCode}
\end{Examples}

