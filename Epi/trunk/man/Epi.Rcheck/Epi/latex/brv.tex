\HeaderA{brv}{Bereavement in an elderly cohort}{brv}
\keyword{datasets}{brv}
\begin{Description}\relax
The \code{brv} data frame has 399 rows and 11 columns.
The data concern the possible effect of marital bereavement on
subsequent mortality. They arose from a survey of the physical and
mental health of a cohort of 75-year-olds in one large general
practice. These data concern mortality up to 1 January, 1990 (although
further follow-up has now taken place).

Subjects included all lived with a living spouse when they entered the
study. There are three distinct groups of such subjects: (1) those in
which both members of the couple were over 75 and therefore included in
the cohort, (2) those whose spouse was below 75 (and was not, therefore,
part of the main cohort study), and (3) those living in larger
households (that is, not just with their spouse).
\end{Description}
\begin{Format}\relax
This data frame contains the following columns:
\Tabular{rl}{
\code{id}: & subject identifier, a numeric vector \\
\code{couple}: & couple identifier, a numeric vector \\
\code{dob}: & date of birth, a date \\
\code{doe}: & date of entry into follow-up study, a date \\
\code{dox}: & date of exit from follow-up study, a date \\
\code{dosp}: & date of death of spouse, a date (if the spouse was still alive
at the end of follow-up, this was coded to January 1, 2000) \\ 
\code{fail}: & status at end of follow-up,
a numeric vector (0=alive,1=dead) \\
\code{group}: & see Description, a numeric vector \\
\code{disab}: & disability score, a numeric vector \\
\code{health}: & perceived health status score, a numeric vector \\
\code{sex}: & a factor with levels
\code{Male} 
\code{Female}
}
\end{Format}
\begin{Source}\relax
Jagger C, and Sutton CJ, Death after Marital Bereavement. Statistics in
Medicine, 10:395-404, 1991. (Data supplied by Carol Jagger).
\end{Source}
\begin{Examples}
\begin{ExampleCode}
data(brv)
\end{ExampleCode}
\end{Examples}

