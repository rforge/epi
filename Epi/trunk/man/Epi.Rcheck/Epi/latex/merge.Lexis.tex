\HeaderA{merge.Lexis}{Merge a Lexis object with a data frame}{merge.Lexis}
\keyword{array}{merge.Lexis}
\keyword{manip}{merge.Lexis}
\begin{Description}\relax
Merge additional variables from a data frame into a Lexis object.
\end{Description}
\begin{Usage}
\begin{verbatim}
## S3 method for class 'Lexis':
merge(x, y, id, by, ...)
\end{verbatim}
\end{Usage}
\begin{Arguments}
\begin{ldescription}
\item[\code{x}] an object of class \code{Lexis}
\item[\code{y}] a data frame
\item[\code{id}] the name of the variable in \code{y} to use for matching
against the variable \code{lex.id} in \code{x}.

\item[\code{by}] if matching is not done by id, a vector of variable names
common to both \code{x} and \code{y}
\item[\code{...}] optional arguments to be passed to \code{merge.data.frame}
\end{ldescription}
\end{Arguments}
\begin{Details}\relax
A \code{Lexis} object can be considered as an augmented data frame
in which some variables are time-dependent variables representing
follow-up. The \code{Lexis} function produces a minimal object
containing only these time-dependent variables.  Additional variables
may be added to a \code{Lexis} object using the \code{merge} method.
\end{Details}
\begin{Value}
A \code{Lexis} object with additional columns taken from the
merged data frame.
\end{Value}
\begin{Note}\relax
The variable given as the \code{by.y} argument must not contain
any duplicate values in the data frame \code{y}.
\end{Note}
\begin{Author}\relax
Martyn Plummer
\end{Author}
\begin{SeeAlso}\relax
\code{\LinkA{merge.data.frame}{merge.data.frame}}, \code{\LinkA{subset.Lexis}{subset.Lexis}}
\end{SeeAlso}

