\HeaderA{ncut}{Function to group a variable in intervals.}{ncut}
\keyword{manip}{ncut}
\begin{Description}\relax
Cuts a continuous variable in intervals. As opposed to \code{cut}
which returns a factor, \code{ncut} returns a numeric variable.
\end{Description}
\begin{Usage}
\begin{verbatim}
ncut(x, breaks, type="left" )
\end{verbatim}
\end{Usage}
\begin{Arguments}
\begin{ldescription}
\item[\code{x}] A numerical vector.
\item[\code{breaks}] Vector of breakpoints. \code{NA} will results for values
below \code{min(x)} if \code{type="left"}, for values
above \code{max(x)} if \code{type="right"} and for values
outside \code{range(x)} if \code{type="mid"}
\item[\code{type}] Character: one of \code{c("left","right","mid")},
indicating whether the left, right or midpoint of the intervals
defined in breaks is returned.
\end{ldescription}
\end{Arguments}
\begin{Details}\relax
The function uses the base function \code{findInterval}.
\end{Details}
\begin{Value}
A numerical vector of the same length as \code{x}.
\end{Value}
\begin{Author}\relax
Bendix Carstensen, Steno Diabetes Center, \email{bxc@steno.dk},
\url{http://www.biostat.ku.dk/~bxc/}, with essential input
from Martyn Plummer, IARC.
\end{Author}
\begin{SeeAlso}\relax
\code{\LinkA{cut}{cut}}, \code{\LinkA{findInterval}{findInterval}}
\end{SeeAlso}
\begin{Examples}
\begin{ExampleCode}
br <- c(-2,0,1,2.5)
x <- c( rnorm( 10 ), br, -3, 3 )
cbind( x, l=ncut( x, breaks=br, type="l" ),
          m=ncut( x, breaks=br, type="m" ),
          r=ncut( x, breaks=br, type="r" ) )[order(x),]
x <- rnorm( 200 )
plot( x, ncut( x, breaks=br, type="l" ), pch=16, col="blue", ylim=range(x) )
abline( 0, 1 )
abline( v=br )
points( x, ncut( x, breaks=br, type="r" ), pch=16, col="red" )
points( x, ncut( x, breaks=br, type="m" ), pch=16, col="green" )
\end{ExampleCode}
\end{Examples}

