\HeaderA{detrend}{Projection of a model matrix on to the orthogonal
complement of a trend.}{detrend}
\keyword{array}{detrend}
\begin{Description}\relax
The columns of the model matrix \code{M} is projected on the
orthogonal complement to the matrix \code{(1,t)}. Orthogonality 
is defined w.r.t. an inner product defined by the weights \code{weight}.
\end{Description}
\begin{Usage}
\begin{verbatim}
  detrend( M, t, weight = rep(1, nrow(M)) )
\end{verbatim}
\end{Usage}
\begin{Arguments}
\begin{ldescription}
\item[\code{M}] A model matrix. 
\item[\code{t}] The trend defining a subspace. A numerical vector of length
\code{nrow(M)} 
\item[\code{weight}] Weights defining the inner product of vectors \code{x}
and \code{y} as \code{sum(x*w*y)}.
A numerical vector of length \code{nrow(M)}, defaults to a vector of
\code{1}s.
\end{ldescription}
\end{Arguments}
\begin{Details}\relax
The functions is intended to be used in parametrization of
age-period-cohort models.
\end{Details}
\begin{Value}
A full-rank matrix with columns orthogonal to \code{(1,t)}.
\end{Value}
\begin{Author}\relax
Bendix Carstensen, Steno Diabetes Center,
\url{http://www.pubhealth.ku.dk/~bxc}, with help from Peter Dalgaard.
\end{Author}
\begin{SeeAlso}\relax
\code{\LinkA{projection.ip}{projection.ip}}
\end{SeeAlso}

