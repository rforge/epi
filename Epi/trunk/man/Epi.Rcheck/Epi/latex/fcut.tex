\HeaderA{fcut}{Cuts follow-up time at multiple failure times.}{fcut}
\keyword{manip}{fcut}
\keyword{datagen}{fcut}
\begin{Description}\relax
This function cuts the follow-up time at multiple failure times
allowing a person to stay at risk between and after the laset
failure. It is aimed at processing of recurrent events. Failure times
outside the interval (\code{enter},\code{exit}) are ignored.
\end{Description}
\begin{Usage}
\begin{verbatim}
fcut( enter, exit, dof, fail = 0,
      data = data.frame(enter, exit),
      Expand = 1:nrow( data ))
\end{verbatim}
\end{Usage}
\begin{Arguments}
\begin{ldescription}
\item[\code{enter}] Date of entry into the study. Numerical vector.
\item[\code{exit}] Date of exit from the study. Numerical vector.
\item[\code{fail}] Failure indicator for the exit date.
\item[\code{dof}] Failure time(s). For multiple failures per individual,
\code{dof} must be a list.
\item[\code{data}] Dataframe of variables to be carried over to the output.
\item[\code{Expand}] Variable identifying original records.
\end{ldescription}
\end{Arguments}
\begin{Value}
A dataframe with the same variables as in \code{data} preceded by the
variables:
\begin{ldescription}
\item[\code{Expand}] Identification of the rows from the input dataframe.
\item[\code{Enter}] Entry date for the interval.
\item[\code{Exit}] Exit date for the interval.
\item[\code{Fail}] Failure indicator for end of the current interval.
\item[\code{n.Fail}] Number of failures prior to the start of the current
interval. Counts all failures given in the list \code{dof},
including those prior to \code{enter}.
\end{ldescription}
\end{Value}
\begin{Author}\relax
Bendix Carstensen, Steno Diabetes Center,
\email{bxc@steno.dk}, \url{www.biostat.ku.dk/~bxc}
\end{Author}
\begin{SeeAlso}\relax
\code{\LinkA{Lexis}{Lexis}},
\code{\LinkA{isec}{isec}},
\code{\LinkA{icut}{icut}},
\code{\LinkA{fcut1}{fcut1}},
\code{\LinkA{ex1}{ex1}}
\end{SeeAlso}
\begin{Examples}
\begin{ExampleCode}
one <- round( runif( 15, 0, 10 ), 1 )
two <- round( runif( 15, 0, 10 ), 1 )
doe <- pmin( one, two )
dox <- pmax( one, two )
# Goofy data rows to test possibly odd behaviour
doe[1:3] <- dox[1:3] <- 8
dox[2] <- 6
dox[3] <- 7.5
# Some failure indicators
fail <- sample( 0:1, 15, replace=TRUE, prob=c(0.7,0.3) )
# Failure times in a list
dof <- sample( c(one,two), 15 )
l.dof <- list( f1=sample( c(one,two), 15 ),
               f2=sample( c(one,two), 15 ),
               f3=sample( c(one,two),15 ) )
# The same, but with events prior to entry removed
lx.dof <- lapply( l.dof, FUN=function(x){ x[x<doe] <- NA ; x } )
# So what have we got
data.frame( doe, dox, fail, l.dof, lx.dof )
# Cut follow-up at event times
fcut( doe, dox, lx.dof, fail, data=data.frame( doe, dox, lx.dof ) )
\end{ExampleCode}
\end{Examples}

