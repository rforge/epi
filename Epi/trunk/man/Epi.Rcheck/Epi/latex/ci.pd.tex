\HeaderA{ci.pd}{Compute confidence limits for a difference of two independent proportions.}{ci.pd}
\keyword{distribution}{ci.pd}
\keyword{htest}{ci.pd}
\begin{Description}\relax
The usual formula for the c.i. of at difference of proportions is
inaccurate. Newcombe has compared 11 methods and method 10 in his
paper looks like a winner. It is implemented here.
\end{Description}
\begin{Usage}
\begin{verbatim}
ci.pd(aa, bb=NULL, cc=NULL, dd=NULL,
     method = "Nc",
      alpha = 0.05, conf.level=0.95,
     digits = 3,
      print = TRUE,
detail.labs = FALSE )
\end{verbatim}
\end{Usage}
\begin{Arguments}
\begin{ldescription}
\item[\code{aa}] Numeric vector of successes in sample 1. Can also be a
matrix or array (see details).
\item[\code{bb}] Successes in sample 2.
\item[\code{cc}] Failures in sample 1.
\item[\code{dd}] Failures in sample 2.
\item[\code{method}] Method to use for calculation of confidence interval, see
"Details".
\item[\code{alpha}] Significance level
\item[\code{conf.level}] Confidence level
\item[\code{print}] Should an account of the two by two table be printed.
\item[\code{digits}] How many digits should the result be rounded to if printed.
\item[\code{detail.labs}] Should the computing of probability differences be
reported in the labels.
\end{ldescription}
\end{Arguments}
\begin{Details}\relax
Implements method 10 from Newcombe(1998) (method="Nc") or from
Agresti \& Caffo(2000) (method="AC").

\code{aa}, \code{bb}, \code{cc} and \code{dd} can be vectors.
If \code{aa} is a matrix, the elements \code{[1:2,1:2]} are used, with
successes \code{aa[,1:2]}. If \code{aa} is a three-way table or array,
the elements \code{aa[1:2,1:2,]} are used.
\end{Details}
\begin{Value}
A matrix with three columns: probability difference, lower and upper
limit. The number of rows equals the length of the vectors  \code{aa},
\code{bb}, \code{cc} and \code{dd} or, if \code{aa} is a 3-way matrix,
\code{dim(aa)[3]}.
\end{Value}
\begin{Author}\relax
Bendix Carstensen, Esa L��r�.
\url{http://www.biostat.ku.dk/~bxc}
\end{Author}
\begin{References}\relax
RG Newcombe: Interval estimation for the difference between
independent proportions. Comparison of eleven methods. Statistics in
Medicine, 17, pp. 873-890, 1998.

A Agresti \& B Caffo: Simple and effective confidence intervals for
proportions and differences of proportions result from adding two
successes and two failures. The American Statistician,
54(4), pp. 280-288, 2000.
\end{References}
\begin{SeeAlso}\relax
\code{\LinkA{twoby2}{twoby2}}, \code{\LinkA{binom.test}{binom.test}}
\end{SeeAlso}
\begin{Examples}
\begin{ExampleCode}
( a <- matrix( sample( 10:40, 4 ), 2, 2 ) )
ci.pd( a )
twoby2( t(a) )
prop.test( t(a) )
( A <- array( sample( 10:40, 20 ), dim=c(2,2,5) ) )
ci.pd( A )
ci.pd( A, detail.labs=TRUE, digits=3 )
\end{ExampleCode}
\end{Examples}

