\HeaderA{Relevel}{Reorder and combine levels of a factor}{Relevel}
\keyword{manip}{Relevel}
\begin{Description}\relax
The levels of a factor are re-ordered so that the levels specified by
\code{ref} is first and the others are moved down. This is useful for
\code{contr.treatment} contrasts which take the first level as the
reference. Levels may also be combined.
\end{Description}
\begin{Usage}
\begin{verbatim}
Relevel(f, ref, first = TRUE, collapse="+" )
\end{verbatim}
\end{Usage}
\begin{Arguments}
\begin{ldescription}
\item[\code{f}] An unordered factor
\item[\code{ref}] The names or numbers of levels to be the first. If \code{ref}
is a list, factor levels mentioned in each list element are
combined. If the list is named the names are used as new factor levels.
\item[\code{first}] Should the levels mentioned in ref come before those not?
\item[\code{collapse}] String used when collapsing factor levels.
\end{ldescription}
\end{Arguments}
\begin{Value}
An unordered factor.
\end{Value}
\begin{Examples}
\begin{ExampleCode}
ff <- factor( sample( letters[1:5], 100, replace=TRUE ) )
table( ff, Relevel( ff, list( AB=1:2, "Dee"=4, c(3,5) ) ) )
table( ff, rr=Relevel( ff, list( 5:4, Z=c("c","a") ), coll="-und-", first=FALSE ) )
\end{ExampleCode}
\end{Examples}

