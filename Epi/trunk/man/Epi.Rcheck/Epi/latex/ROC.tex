\HeaderA{ROC}{Function to compute and draw ROC-curves.}{ROC}
\keyword{manip}{ROC}
\keyword{htest}{ROC}
\begin{Description}\relax
Computes sensitivity, specificity and positive and negative predictive
values for a test based on dichotomizing along the variable
\code{test}, for prediction of \code{stat}. Alternatively a
model formula may given, in which case the the linear predictor is the
test variable and the response is taken as the true status variable.
Plots curves of these and a ROC-curve.
\end{Description}
\begin{Usage}
\begin{verbatim}
ROC( test = NULL,
     stat = NULL,
     form = NULL,
     plot = c("sp", "ROC"),
       PS = is.null(test),    
       PV = TRUE,             
       MX = TRUE,             
       MI = TRUE,             
      AUC = TRUE,             
     grid = seq(0,100,10),
 col.grid = gray( 0.9 ),
     cuts = NULL,
      lwd = 2,
     data = parent.frame(), 
      ... )
\end{verbatim}
\end{Usage}
\begin{Arguments}
\begin{ldescription}
\item[\code{test}] Numerical variable used for prediction. 
\item[\code{stat}] Logical variable of true status. 
\item[\code{form}] Formula used in a logistic regression. If this is given,
\code{test} and \code{stat} are ignored. If not given then
both \code{test} and \code{stat} must be supplied. 
\item[\code{plot}] Character variable. If "sp", the a plot of sensitivity,
specificity and predictive values against test is produced, if "ROC" a
ROC-curve is plotted. Both may be given.
\item[\code{PS}] logical, if TRUE the x-axis in the
plot "ps"-plot is the the predicted probability for
\code{stat}==TRUE, otherwise it is the scale of \code{test} if this
is given otherwise the scale of the linear predictor from the
logistic regression.
\item[\code{PV}] Should sensitivity, specificity and
predictive values at the optimal cutpoint be given on the ROC plot? 
\item[\code{MX}] Should the ``optimal cutpoint'' (i.e. where sens+spec is
maximal) be indicated on the ROC curve?
\item[\code{MI}] Should model summary from the logistic
regression model be printed in the plot?
\item[\code{AUC}] Should the area under the curve (AUC) be printed in the ROC
plot?
\item[\code{grid}] Numeric or logical. If FALSE no background grid is
drawn. Otherwise a grid is drawn on both axes at \code{grid} percent.
\item[\code{col.grid}] Colour of the grid lines drawn.
\item[\code{cuts}] Points on the test-scale to be annotated on the
ROC-curve. 
\item[\code{lwd}] Thickness of the curves
\item[\code{data}] Data frame in which to interpret the variables.
\item[\code{...}] Additional arguments for the plotting of the
ROC-curve. Passed on to \code{plot}
\end{ldescription}
\end{Arguments}
\begin{Value}
A list with two components:
\begin{ldescription}
\item[\code{res}] dataframe with variables sn, sp, pvp, pvn and fv. The
latter is the unique values of test (for PS==FALSE ) or linear predictor from the
logistic regression
\item[\code{lr}] glm object with the logistic regression result used for
construction of the ROC curve
\end{ldescription}

0, 1 or 2 plots are produced according to the setting of \code{plot}.
\end{Value}
\begin{Author}\relax
Bendix Carstensen, Steno Diabetes Center \& University of
Copenhagen,
\url{http://www.biostat.ku.dk/~bxc}
\end{Author}
\begin{References}\relax
\end{References}
\begin{Examples}
\begin{ExampleCode}
x <- rnorm( 100 )
z <- rnorm( 100 )
w <- rnorm( 100 )
tigol <- function( x ) 1 - ( 1 + exp( x ) )^(-1)
y <- rbinom( 100, 1, tigol( 0.3 + 3*x + 5*z + 7*w ) )
ROC( form = y ~ x + z, plot="ROC" )
\end{ExampleCode}
\end{Examples}

