\HeaderA{plot.Lexis}{Lexis diagrams}{plot.Lexis}
\aliasA{lines.Lexis}{plot.Lexis}{lines.Lexis}
\aliasA{points.Lexis}{plot.Lexis}{points.Lexis}
\keyword{hplot}{plot.Lexis}
\keyword{aplot}{plot.Lexis}
\begin{Description}\relax
The follow-up histories represented by a Lexis object can be plotted
using one or two dimensions.  The two dimensional plot is a Lexis
diagram showing follow-up time simultaneously on two time scales.
\end{Description}
\begin{Usage}
\begin{verbatim}
## S3 method for class 'Lexis':
plot(x, time.scale = NULL, breaks="lightgray", ...)
## S3 method for class 'Lexis':
points(x, time.scale = NULL, ...)
## S3 method for class 'Lexis':
lines(x, time.scale = NULL, ...)
\end{verbatim}
\end{Usage}
\begin{Arguments}
\begin{ldescription}
\item[\code{x}] An object of class \code{Lexis}
\item[\code{time.scale}] A vector of length 1 or 2 giving the time scales to
be plotted either by name or numerical order
\item[\code{breaks}] a string giving the colour of grid lines to be drawn
when plotting a split Lexis object. Grid lines can be suppressed by
supplying the value \code{NULL} to the \code{breaks} argument
\item[\code{...}] Further graphical parameters to be passed to the plotting
methods. Grids can be drawn using the followin parameters in \code{plot}:
\Itemize{
\item \code{grid} If logical, a background grid is set up
using the axis ticks. If a list, the first component is used as
positions for the vertical lines and the last as positions for the
horizontal. If a nunerical vector, grids on both axes are set up
using the distance between the numbers.
\item \code{col.grid="lightgray"} Color of the background grid.
\item \code{lty.grid=2} Line type for the grid.
\item \code{coh.grid=FALSE} Should a 45 degree grid be plotted?}

\end{ldescription}
\end{Arguments}
\begin{Details}\relax
The plot method for \code{Lexis} objects traces ``life lines'' from
the start to the end of follow-up.  The \code{points} method plots
points at the end of the life lines.

If \code{time.scale} is of length 1, the life lines are drawn
horizontally, with the time scale on the X axis and the id value on the Y
axis. If \code{time.scale} is of length 2, a Lexis diagram is
produced, with diagonal life lines plotted against both time scales
simultaneously.

If \code{lex} has been split along one of the time axes by a call to
\code{splitLexis}, then vertical or horizontal grid lines are plotted
at the break points.
\end{Details}
\begin{Author}\relax
Martyn Plummer
\end{Author}
\begin{SeeAlso}\relax
\code{\LinkA{Lexis}{Lexis}}, \code{\LinkA{splitLexis}{splitLexis}}
\end{SeeAlso}

