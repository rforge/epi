\HeaderA{effx}{Function to calculate effects}{effx}
\keyword{models}{effx}
\keyword{regression}{effx}
\begin{Description}\relax
The function calculates the effects of an exposure on a response,
possibly stratified by a stratifying variable, and/or controlled for one
of more confounding variables.
\end{Description}
\begin{Usage}
\begin{verbatim}
effx( response, type = "metric",
                 fup = NULL,     
            exposure,     
              strata = NULL,  
             control = NULL,
             weights = NULL,
               alpha = 0.05,   
                base = 1,             
              sigdig = 3,     
                data = NULL )    
\end{verbatim}
\end{Usage}
\begin{Arguments}
\begin{ldescription}
\item[\code{response}] The \code{response} variable - must be numeric
\item[\code{type}] The type of response\code{type} - must be one of "metric",
"binary", "failure", or "count"
\item[\code{fup}] The \code{fup} variable contains the follow-up time for a
failure response
\item[\code{exposure}] The \code{exposure} variable can be numeric or a factor
\item[\code{strata}] The \code{strata} stratifying variable - must be a factor
\item[\code{control}] The \code{control} variable(s) - these are passed as a
list of there are more than one.
\item[\code{weights}] Weights
\item[\code{base}] Baseline for the effects of a categorical exposure, default 1
\item[\code{sigdig}] Number of significant digits for the effects, default 3
\item[\code{alpha}] 1 - confidence level
\item[\code{data}] \code{data} refers to the data used to evaluate the function
\end{ldescription}
\end{Arguments}
\begin{Details}\relax
The function is a wrapper for glm. Effects are calculated as
differences in means for a metric response, odds ratios for a binary
response, and rate ratios for a failure or count response.  

The k-1 effects for a categorical exposure with k levels are relative
to a baseline which, by default, is the first level. The effect of a
metric (quantitative) exposure is calculated per unit of exposure. 

The exposure variable can be numeric or a factor, but if it is an
ordered factor the order will be ignored.
\end{Details}
\begin{Value}
\begin{ldescription}
\item[\code{comp1 }] Effects of exposure
\item[\code{comp2 }] Tests of significance
\end{ldescription}
\end{Value}
\begin{Section}{Warning}
The function attaches the frame specified in the
\code{data=} argument, so there is a possibility that a variable in the
Global environment is masked by the attached frame. Watch out for this
warning in the output!
\end{Section}
\begin{Author}\relax
Michael Hills
\end{Author}
\begin{References}\relax
www.mhills.pwp.blueyonder.co.uk
\end{References}
\begin{Examples}
\begin{ExampleCode}
library(Epi)
data(births)
births$hyp <- factor(births$hyp,labels=c("normal","hyper"))
births$sex <- factor(births$sex,labels=c("M","F"))

# bweight is the birth weight of the baby in gms, and is a metric
# response (the default) 

# effect of hypertension on birth weight
effx(bweight,exposure=hyp,data=births) 
# effect of hypertension on birth weight stratified by sex
effx(bweight,exposure=hyp,strata=sex,data=births) 
# effect of hypertension on birth weight controlled for sex
effx(bweight,exposure=hyp,control=sex,data=births) 
# effect of gestation time on birth weight
effx(bweight,exposure=gestwks,data=births) 
# effect of gestation time on birth weight stratified by sex
effx(bweight,exposure=gestwks,strata=sex,data=births) 
# effect of gestation time on birth weight controlled for sex
effx(bweight,exposure=gestwks,control=sex,data=births) 

# lowbw is a binary response coded 1 for low birth weight and 0 otherwise
# effect of hypertension on low birth weight
effx(lowbw,type="binary",exposure=hyp,data=births)
# etc.
\end{ExampleCode}
\end{Examples}

