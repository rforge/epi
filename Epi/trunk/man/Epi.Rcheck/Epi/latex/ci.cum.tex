\HeaderA{ci.cum}{Compute cumulative sum of estimates.}{ci.cum}
\keyword{models}{ci.cum}
\keyword{regression}{ci.cum}
\begin{Description}\relax
Computes the cumulative sum of parameter functions and the
standard error of it. Optionally the exponential is applied to the
parameter functions before it is cumulated.
\end{Description}
\begin{Usage}
\begin{verbatim}
ci.cum( obj,
    ctr.mat = NULL,
     subset = NULL,
       intl = 1,
      alpha = 0.05,
        Exp = TRUE )
\end{verbatim}
\end{Usage}
\begin{Arguments}
\begin{ldescription}
\item[\code{obj}] A model object. 
\item[\code{ctr.mat}] Contrast matrix defining the parameter functions from
the parameters of the model. 
\item[\code{subset}] Subset of the parameters of the model to which
\code{ctr.mat} should be applied. 
\item[\code{intl}] Interval length for the cumulation. Either a constant or
a numerical vector of length \code{nrow(ctr.mat)}. 
\item[\code{alpha}] Significance level used when computing confidence limits. 
\item[\code{Exp}] Should the parameter function be exponentiated before it is
cumulated? 
\end{ldescription}
\end{Arguments}
\begin{Details}\relax
The purpose of this function is to compute cumulative rate based on a
model for the rates. If the model is a multiplicative model for the
rates, the purpose of \code{ctr.mat} is to return a vector of rates or
log-rates when applied to the coefficients of the model. If log-rates
are returned, the they should be exponentiated before cumulated, and
the variances computed accordingly. Since log-linear models are the most
common the \code{Exp} parameter defaults to TRUE.
\end{Details}
\begin{Value}
A matrix with 4 columns: Estimate, lower and upper c.i. and standard
error. The numebr of rows equals \code{nrow(ctr.mat)} if \code{zero} if
FALSE, otherwise \code{nrow(ctr.mat)+1}, because an initial row of 0s
is added.
\end{Value}
\begin{Author}\relax
Bendix Carstensen,
\url{http://www.pubhealth.ku.dk/~bxc}
\end{Author}
\begin{SeeAlso}\relax
See also \code{\LinkA{ci.lin}{ci.lin}}
\end{SeeAlso}
\begin{Examples}
\begin{ExampleCode}
# Packages required for this example
library( splines )
library( survival )
data( lung )
par( mfrow=c(1,2) )

# Plot the Kaplan-meier-estimator
#
plot( survfit( Surv( time, status==2 ), data=lung ) )

# Cut the follow-up every 10 days
#
dx <- W.Lexis( exit=time, fail=(status==2), breaks=seq(0,1100,10), data=lung )
str( dx )

# Fit a Poisson model with a natural spline for the effect of time.
# Note that "Time" is the left endpoint of the interval in the dataframe dx.
#
MM <- ns( dx$Time, knots=c(50,100,200,400,700), intercept=TRUE )
mp <- glm( Fail ~ MM - 1 + offset(log(Exit-Entry)),
                  family=poisson, data=dx, eps=10^-8, maxit=25 )

# Contrast matrix to extract effects, i.e. matrix to multiply with the
# coefficients to produce the log-rates: unique rows of MM, in time order.
#
T.pt <- sort( unique( dx$Time ) )
T.wh <- match( T.pt, dx$Time )
Lambda <- ci.cum( mp, ctr.mat=MM[T.wh,], intl=diff(c(0,T.pt)) )

# Put the estimated survival function on top of the KM-estimator
#
matlines( c(0,T.pt[-1]), exp(-Lambda[,1:3]), lwd=c(3,1,1), lty=1, col="Red" )

# Extract and plot the fitted intensity function
#
lambda <- ci.lin( mp, ctr.mat=MM[T.wh,], Exp=TRUE )
matplot( T.pt, lambda[,5:7]*10^3, type="l", lwd=c(3,1,1), col="black", lty=1,
         log="y", ylim=c(0.2,20) )
\end{ExampleCode}
\end{Examples}

