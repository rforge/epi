\HeaderA{hivDK}{hivDK: seroconversion in a cohort of Danish men}{hivDK}
\keyword{datasets}{hivDK}
\begin{Description}\relax
Data from a survey of HIV-positivity of a cohort of Danish
men followed by regular tests from 1983 to 1989.
\end{Description}
\begin{Usage}
\begin{verbatim}
  data(hivDK)
  \end{verbatim}
\end{Usage}
\begin{Format}\relax
A data frame with 297 observations on the following 7 variables.
\describe{
\item[\code{id}] ID of the person
\item[\code{entry}] Date of entry to the study. Date variable.
\item[\code{well}] Date last seen seronegative. Date variable.
\item[\code{ill}] Date first seen seroconverted. Date variable.
\item[\code{bth}] Year of birth minus 1950.
\item[\code{pyr}] Annual number of sexual partners.
\item[\code{us}] Indicator of wheter the person has visited the USA.
}
\end{Format}
\begin{Source}\relax
Mads Melbye, Statens Seruminstitut.
\end{Source}
\begin{References}\relax
Becker N.G. and Melbye M.: Use of a log-linear model to
compute the empirical survival curve from interval-censored data,
with application to data on tests for HIV-positivity, Australian
Journal of Statistics, 33, 125--133, 1990.

Melbye M., Biggar R.J., Ebbesen P., Sarngadharan M.G., Weiss
S.H., Gallo R.C. and Blattner W.A.: Seroepidemiology of HTLV-III
antibody in Danish homosexual men: prevalence, transmission and
disease outcome. British Medical Journal, 289, 573--575, 1984.
\end{References}
\begin{Examples}
\begin{ExampleCode}
  data(hivDK)
  str(hivDK) 
  \end{ExampleCode}
\end{Examples}

