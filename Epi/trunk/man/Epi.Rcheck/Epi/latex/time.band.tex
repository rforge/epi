\HeaderA{time.band}{Extract time band data from a split Lexis object}{time.band}
\aliasA{breaks}{time.band}{breaks}
\aliasA{timeBand}{time.band}{timeBand}
\keyword{attribute}{time.band}
\begin{Description}\relax
The break points of a \code{Lexis} object (created by a call to
\code{splitLexis}) divide the follow-up intervals into time bands
along a given time scale. The \code{breaks} function returns
the break points, for a given time scale, and the \code{timeBand}
classifies each row into one of the time bands.
\end{Description}
\begin{Usage}
\begin{verbatim}
timeBand(lex, time.scale, type)
breaks(lex, time.scale)
\end{verbatim}
\end{Usage}
\begin{Arguments}
\begin{ldescription}
\item[\code{lex}] an object of class \code{Lexis}
\item[\code{time.scale}] a character or integer vector of length 1
identifying the time scale of interest
\item[\code{type}] a string that determines how the time bands are labelled.
See Details below
\end{ldescription}
\end{Arguments}
\begin{Details}\relax
Time bands may be labelled in various ways according to the
\code{type} argument. The permitted values of the \code{type}
argument, and the corresponding return values are:
\item["factor"] a factor (unordered) with labels "(left,right]"
\item["integer"] a numeric vector with integer codes starting from 0
\item["left"] the left-hand limit of the time band
\item["middle"] the middle of the time band
\item["right"] the right-hand limit of the time band
\end{Details}
\begin{Value}
The \code{breaks} function returns a vector of break points
for the \code{Lexis} object, or NULL if no break points have been
defined by a call to \code{splitLexis}.  The \code{timeBand}
function returns a numeric vector or factor, depending on the value
of the \code{type} argument.
\end{Value}
\begin{Note}\relax
A newly created \code{Lexis} object has no break points defined.
In this case, \code{breaks} will return NULL, and
\code{timeBand} will return a degenerate factor with one level
labelled "(-Inf,Inf]".
\end{Note}
\begin{Author}\relax
Martyn Plummer
\end{Author}
\begin{SeeAlso}\relax
\code{\LinkA{Lexis}{Lexis}}
\end{SeeAlso}

