\HeaderA{start.Lexis}{Time series methods for Lexis objects}{start.Lexis}
\aliasA{deltat.Lexis}{start.Lexis}{deltat.Lexis}
\aliasA{entry}{start.Lexis}{entry}
\aliasA{exit}{start.Lexis}{exit}
\aliasA{status}{start.Lexis}{status}
\keyword{ts}{start.Lexis}
\begin{Description}\relax
Extract the entry time, exit time, status, or duration of follow-up from a
\code{Lexis} object.
\end{Description}
\begin{Usage}
\begin{verbatim}
entry(x, time.scale = NULL)
exit(x, time.scale = NULL)
status(x, at=c("exit","entry"))
## S3 method for class 'Lexis':
deltat(x, ...)
\end{verbatim}
\end{Usage}
\begin{Arguments}
\begin{ldescription}
\item[\code{x}] an object of class \code{Lexis}
\item[\code{time.scale}] a string or integer indicating the time scale. If
omitted, the first time scale is used as a default
\item[\code{at}] string indicating the time point at which status is to be measured
\item[\code{...}] optional arguments to the \code{deltat} method
\end{ldescription}
\end{Arguments}
\begin{Value}
The \code{entry} and \code{exit} functions return a vector containing
the entry times and exit times, respectively, on the requested time
scale. The \code{status} function returns the status either at entry
or (by default) at exit, and the \code{deltat} method returns the length
of the follow-up interval.
\end{Value}
\begin{Author}\relax
Martyn Plummer
\end{Author}
\begin{SeeAlso}\relax
\code{\LinkA{Lexis}{Lexis}}
\end{SeeAlso}

