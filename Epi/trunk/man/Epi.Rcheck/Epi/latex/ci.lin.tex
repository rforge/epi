\HeaderA{ci.lin}{Compute linear functions of parameters with s.e.}{ci.lin}
\keyword{models}{ci.lin}
\keyword{regression}{ci.lin}
\begin{Description}\relax
For a given model object the function computes a linear function of
the parameters and the corresponding standard errors, p-values and
confidence intervals.
\end{Description}
\begin{Usage}
\begin{verbatim}
ci.lin( obj,
    ctr.mat = NULL,
     subset = NULL,
      diffs = FALSE,
       fnam = !diffs,
       vcov = FALSE,
      alpha = 0.05,
        Exp = FALSE )
\end{verbatim}
\end{Usage}
\begin{Arguments}
\begin{ldescription}
\item[\code{obj}] A model object (of class \code{lm}, \code{glm}, \code{lme},
\code{coxph} or \code{polr}). 

\item[\code{ctr.mat}] Contrast matrix to be multiplied to the parameter
vector, i.e. the desired linear function of the parameters.
\item[\code{subset}] The subset of the parameters to be used. If given as a
character vector, the elements are in turn matched against the
parameter names (using \code{grep}) to find the subset. Repeat
parameters may result from using a character vector. This is
considered a facility.
\item[\code{diffs}] If TRUE, all differences between parameters
in the subset are computed. \code{ctr.mat} is ignored. If \code{obj}
inherits from \code{lm}, and \code{subset} is given as a string
\code{subset} is used to search among the factors in the model and
differences of all factor levels for the first match are shown.
If \code{subset} does not match any of the factors in the model, all
pairwise differences between parameters matching are returned.
\item[\code{fnam}] Should the common part of the parameter names be included
with the annotation of contrasts? Ignored if \code{diffs==T}. If a
sting is supplied this will be prefixed to the labels.
\item[\code{vcov}] Should the covariance matrix of the set of parameters be
returned? If this is set, \code{Exp} is ignored.
\item[\code{alpha}] Significance level for the confidence intervals.
\item[\code{Exp}] If \code{TRUE} columns 5:6 are replaced with exp( columns 1,5,6 ).
\end{ldescription}
\end{Arguments}
\begin{Value}
A matrix with number of rows and rownames as \code{ctr.mat}. The
columns are Estimate, Std.Err, z, P, 2.5\% and 97.5\%.
If \code{vcov=TRUE} a list with components \code{est}, the desired
functional of the parameters and \code{vcov}, the variance
covariance matrix of this, is returned but not printed.
If code{Exp==TRUE} the confidence intervals for the parameters are
replaced with three columns: exp(estimate,c.i.).
\end{Value}
\begin{Author}\relax
Bendix Carstensen,
\url{http://www.pubhealth.ku.dk/~bxc}
\end{Author}
\begin{Examples}
\begin{ExampleCode}
# Bogus data:
f <- factor( sample( letters[1:5], 200, replace=TRUE ) )
g <- factor( sample( letters[1:3], 200, replace=TRUE ) )
x <- rnorm( 200 )
y <- 7 + as.integer( f ) * 3 + 2 * x + 1.7 * rnorm( 200 )

# Fit a simple model:
mm <- lm( y ~ x + f + g )
ci.lin( mm ) 
ci.lin( mm, subset=3:6, diff=TRUE, fnam=FALSE )
ci.lin( mm, subset=3:6, diff=TRUE, fnam=TRUE )
ci.lin( mm, subset="f", diff=TRUE, fnam="f levels:" )
print( ci.lin( mm, subset="g", diff=TRUE, fnam="gee!:", vcov=TRUE ) )

# Use character defined subset to get ALL contrasts:
ci.lin( mm, subset="f", diff=TRUE )
\end{ExampleCode}
\end{Examples}

