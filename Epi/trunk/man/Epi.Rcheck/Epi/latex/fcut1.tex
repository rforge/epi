\HeaderA{fcut1}{Cut follow-up time at a failure time.}{fcut1}
\keyword{manip}{fcut1}
\keyword{datagen}{fcut1}
\begin{Description}\relax
This function cuts the follow-up time at a failure allowing a person
to stay at risk after the failure. It is aimed at processing of
recurrent events. Failure times outside the interval
(\code{enter},\code{exit}) are ignored.
\end{Description}
\begin{Usage}
\begin{verbatim}
fcut1( enter, exit, fail, dof, fail.value = 1, 
         data = data.frame(enter, exit, fail, dof),
       Expand = 1:nrow(data) )
\end{verbatim}
\end{Usage}
\begin{Arguments}
\begin{ldescription}
\item[\code{enter}] Date of entry into the study. Numerical vector.
\item[\code{exit}] Date of exit from the study. Numerical vector.
\item[\code{fail}] Failure indicator for the exit date.
\item[\code{dof}] Date of failure. Numerical vector. Missing for persons who
do not have an event.
\item[\code{fail.value}] Value for failure indicator \code{Fail} at the date
\code{dof}.
\item[\code{data}] Dataframe of variables to be carried unchanged to the
output.
\item[\code{Expand}] Variable identifying original records.
\end{ldescription}
\end{Arguments}
\begin{Value}
A dataframe with the same columns as in \code{data}, preceded by the columns:
\begin{ldescription}
\item[\code{Expand}] Identification of the rows from the input dataframe.
\item[\code{Enter}] Entry date for the interval.
\item[\code{Exit}] Exit date for the interval.
\item[\code{Fail}] Failure indicator for the interval.
\end{ldescription}
\end{Value}
\begin{Author}\relax
Bendix Carstensen, Steno Diabetes Center,
\email{bxc@steno.dk}, \url{www.biostat.ku.dk/~bxc}
\end{Author}
\begin{SeeAlso}\relax
\code{\LinkA{Lexis}{Lexis}},
\code{\LinkA{isec}{isec}},
\code{\LinkA{fcut1}{fcut1}},
\code{\LinkA{fcut}{fcut}},
\code{\LinkA{ex1}{ex1}}
\end{SeeAlso}
\begin{Examples}
\begin{ExampleCode}
one <- round( runif( 15, 0, 10 ), 1 )
two <- round( runif( 15, 0, 10 ), 1 )
doe <- pmin( one, two )
dox <- pmax( one, two )
# Goofy data rows to test possibly odd behaviour
doe[1:3] <- dox[1:3] <- 8
dox[2] <- 6
dox[3] <- 7.5
# Some failure indicators and failure times
fail <- sample( 0:1, 15, replace=TRUE, prob=c(0.7,0.3) )
dof <- sample( c(one,two), 15 )
# So what have we got
data.frame( doe, dox, fail, dof )
# Cut follow-up at dof
fcut1( doe, dox, fail, dof )
\end{ExampleCode}
\end{Examples}

