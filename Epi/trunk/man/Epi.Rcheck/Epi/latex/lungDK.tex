\HeaderA{lungDK}{Male lung cancer incidence in Denmark}{lungDK}
\keyword{datasets}{lungDK}
\begin{Description}\relax
Male lung cancer cases and population riks time in Denmark, for the
period 1943--1992 in ages 40--89.
\end{Description}
\begin{Usage}
\begin{verbatim}data(lungDK)\end{verbatim}
\end{Usage}
\begin{Format}\relax
A data frame with 220 observations on the following 9 variables.
\Tabular{rl}{
\code{A5}: & Left end point of the age interval, a numeric vector. \\
\code{P5}: & Left enpoint of the period interval, a numeric vector. \\
\code{C5}: & Left enpoint of the birth cohort interval, a numeric vector. \\
\code{up}: & Indicator of upper trianges of each age by period
rectangle in the Lexis diagram. (\code{up=(P5-A5-C5)/5}). \\
\code{Ax}: & The mean age of diagnois (at risk) in the triangle. \\
\code{Px}: & The mean date of diagnosis (at risk) in the triangle. \\
\code{Cx}: & The mean date of birth in the triangle, a numeric vector. \\
\code{D}: & Number of diagnosed cases of male lung cancer. \\
\code{Y}: & Risk time in the male population, person-years. \\
}
\end{Format}
\begin{Details}\relax
Cases and person-years are tabulated by age and date of
diagnosis (period) as well as date of birth (cohort) in 5-year
classes. Each observation in the dataframe correponds to a triangle in
a Lexis diagram. Triangles are classified by age and date of
diagnosis, period of diagnosis and date of birth, all in 5-year
groupings.
\end{Details}
\begin{Source}\relax
The Danish Cancer Registry and Statistics Denmark.
\end{Source}
\begin{References}\relax
For a more thorough exposition of statistical inference in the Lexis
diagram, see: \url{http://staff.pubhealth.ku.dk/~bxc/APC/notes.pdf}
\end{References}
\begin{Examples}
\begin{ExampleCode}
data( lungDK )
# Draw a Lexis diagram and show the number of cases in it.
attach( lungDK )
Lexis.diagram( age=c(40,90), date=c(1943,1993), coh.grid=TRUE )
text( Px, Ax, paste( D ), cex=0.7 )
\end{ExampleCode}
\end{Examples}

