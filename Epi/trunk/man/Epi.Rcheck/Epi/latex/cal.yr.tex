\HeaderA{cal.yr}{Functions to convert character, factor and various date objects into a number,
and vice versa.}{cal.yr}
\aliasA{as.Date.cal.yr}{cal.yr}{as.Date.cal.yr}
\aliasA{as.Date.numeric}{cal.yr}{as.Date.numeric}
\keyword{manip}{cal.yr}
\keyword{chron}{cal.yr}
\begin{Description}\relax
Dates are converted to a numerical value, giving the calendar year as
a fractional number. 1 January 1970 is converted to 1970.0, and other
dates are converted by assuming that years are all 365.25 days long,
so inaccuracies may arise, for example, 1 Jan 2000 is converted to
1999.999. Differences between converted values will be 1/365.25 of the 
difference between corresponding \code{\LinkA{Date}{Date}} objects.
\end{Description}
\begin{Usage}
\begin{verbatim}
  cal.yr(x, format)
  as.Date.cal.yr( x, ... )
  as.Date.numeric( x, ..., unit="d" )
\end{verbatim}
\end{Usage}
\begin{Arguments}
\begin{ldescription}
\item[\code{x}] A factor or character vector, representing a date in format
\code{format}, or an object of class
\code{\LinkA{Date}{Date}},
\code{\LinkA{POSIXlt}{POSIXlt}},
\code{\LinkA{POSIXct}{POSIXct}},
\code{\LinkA{date}{date}},
\code{dates} or
\code{chron} (the latter two requires the \code{chron} package).
\item[\code{format}] Format of the date values if \code{x} is factor or character
\item[\code{unit}] Which units are the date measured in, \code{"y"} for years,
\code{"d"} for days.
\item[\code{...}] Arguments passed on from other methods.
\end{ldescription}
\end{Arguments}
\begin{Value}
\code{cal.yr} returns a numerical vector of the same length as
\code{x}, of class \code{c("cal.yr","numeric")}.

\code{as.Date.cal.yr} and \code{as.Date.numeric}
return \code{\LinkA{Date}{Date}} objects.
\end{Value}
\begin{Author}\relax
Bendix Carstensen, Steno Diabetes Center \& Dept. of Biostatistics,
University of Copenhagen, \email{bxc@steno.dk},
\url{http://www.pubhealth.ku.dk/~bxc}
\end{Author}
\begin{SeeAlso}\relax
\code{\LinkA{DateTimeClasses}{DateTimeClasses}},
\code{\LinkA{Date}{Date}}
\end{SeeAlso}
\begin{Examples}
\begin{ExampleCode}
 # Charcter vector of dates:
 birth <- c("14/07/1852","01/04/1954","10/06/1987","16/05/1990",
            "01/01/1996","01/01/1997","01/01/1998","01/01/1999")
 # Proper conversion to class "Date":
 birth.dat <- as.Date( birth, format="%d/%m/%Y" )
 # Converson of character to class "cal.yr"
 bt.yr <- cal.yr( birth, format="%d/%m/%Y" )
 # Back to class "Date":
 bt.dat <- as.Date( bt.yr )
 # Numerical calculation of days since 1.1.1970:
 days <- Days <- (bt.yr-1970)*365.25
 # Blunt assignment of class:
 class( Days ) <- "Date"
 # Then data.frame() to get readable output of results:
 data.frame( birth, birth.dat, bt.yr, bt.dat, days, Days, round(Days) )
\end{ExampleCode}
\end{Examples}

